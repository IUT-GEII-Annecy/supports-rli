% ==================== NIVEAU 3 — VARIABLES / TYPES (10) ====================

\element{niveau3}{
\begin{question}{n3-trace1}
Donner les valeurs finales de \lstinline|a|, \lstinline|b|, \lstinline|c|.
\begin{lstlisting}[language=C]
int a=5, b=10, c=0;
c = a + b;
b = c - a;
a = a + 1;
c = a + b + c;
\end{lstlisting}
\begin{reponses}
  \bonne{a=6, b=10, c=31}
  \mauvaise{a=6, b=5, c=21}
  \mauvaise{a=5, b=10, c=15}
  \mauvaise{a=16, b=5, c=21}
  \mauvaise{a=0, b=0, c=0}
  \mauvaise{Erreur}
\end{reponses}
\end{question}
}

% \element{niveau3}{
% \begin{question}{n3-trace2}
% Donner les valeurs finales de \lstinline|x|, \lstinline|y|, \lstinline|z|.
% \begin{lstlisting}[language=C]
% int x=2,y=3,z=4;
% x=y+z;
% y=x-z;
% z=x+y+z;
% \end{lstlisting}
% \begin{reponses}
%   \bonne{x=7,y=3,z=14}
%   \mauvaise{x=5,y=2,z=7}
%   \mauvaise{x=7,y=1,z=8}
%   \mauvaise{x=9,y=3,z=12}
%   \mauvaise{x=0,y=0,z=0}
%   \mauvaise{Erreur}
% \end{reponses}
% \end{question}
% }

% \element{niveau3}{
% \begin{question}{n3-open-struct}
% Proposer une organisation en C pour stocker les informations de plusieurs étudiants (prénom, âge, moyenne).  
% Comparer l’utilisation de tableaux parallèles et de \lstinline|struct|.
% \evaluationProfLignes{4}{4}
% \end{question}
% }

% \element{niveau3}{
% \begin{question}{n3-open-doublefloat}
% Expliquer en 3 phrases la différence entre \lstinline|float| et \lstinline|double| (taille, précision, plage).  
% Illustrer avec un exemple de calcul où le choix du type change le résultat.
% \evaluationProfLignes{3}{3}
% \end{question}
% }

% \element{niveau3}{
% \begin{question}{n3-cast}
% Que vaut \lstinline|(int)(10/4 + 10.0/4)| ?
% \begin{reponses}
%   \bonne{7}
%   \mauvaise{5}
%   \mauvaise{2}
%   \mauvaise{2.5}
%   \mauvaise{4}
%   \mauvaise{Erreur}
% \end{reponses}
% \end{question}
% }

% \element{niveau3}{
% \begin{question}{n3-open-debordement}
% Expliquer ce qui se passe pour :  
% \lstinline|int x = 2000000000 + 2000000000;|  
% Donner un exemple d’impact concret (finance, capteur…).
% \evaluationProfLignes{3}{3}
% \end{question}
% }

% \element{niveau3}{
% \begin{question}{n3-affichage-mix}
% Que va afficher ce code ?
% \begin{lstlisting}[language=C]
% int i=7, j=3;
% float u=7, v=3;
% i = i + j/2;
% u = u / v;
% printf("%d %f\n", i, u);
% \end{lstlisting}
% \begin{reponses}
%   \bonne{8 ; 2.333333}
%   \mauvaise{7 ; 2.333333}
%   \mauvaise{8 ; 2}
%   \mauvaise{7 ; 2}
%   \mauvaise{Erreur}
%   \mauvaise{9 ; 2.333333}
% \end{reponses}
% \end{question}
% }

% \element{niveau3}{
% \begin{question}{n3-open-erreurs}
% Pourquoi \lstinline|0.1+0.1+0.1| peut-il donner \lstinline|0.30000000000000004| en C ?  
% Expliquer en deux phrases.
% \evaluationProfLignes{2}{2}
% \end{question}
% }

% \element{niveau3}{
% \begin{question}{n3-modulo}
% Que vaut \lstinline|(10*3)%7| ?
% \begin{reponses}
%   \bonne{2}
%   \mauvaise{1}
%   \mauvaise{0}
%   \mauvaise{3}
%   \mauvaise{7}
%   \mauvaise{Erreur}
% \end{reponses}
% \end{question}
% }

% \element{niveau3}{
% \begin{question}{n3-open-somme}
% Écrire un code en C qui calcule la somme des 100 premiers entiers (\(1+2+…+100\)).  
% Proposer une amélioration en utilisant la formule mathématique.
% \evaluationProfLignes{4}{3}
% \end{question}
% }

% % ==================== NIVEAU 3 — CONDITIONS (10) ====================

% \element{niveau3}{
% \begin{question}{n3-cond-tranche}
% Ajouter la règle : gratuit si âge multiple de 10, sinon barème classique.  
% Quelle structure est correcte ?
% \begin{reponses}
%   \bonne{\lstinline[language=C]|if(age%10==0) {...} else if(age<12)...|}
%   \mauvaise{\lstinline[language=C]|if(age/10==0)...|}
%   \mauvaise{\lstinline[language=C]|if(age*10==0)...|}
%   \mauvaise{Toujours gratuit}
%   \mauvaise{Jamais gratuit}
%   \mauvaise{Erreur}
% \end{reponses}
% \end{question}
% }

% \element{niveau3}{
% \begin{question}{n3-open-cond-imbrique}
% Écrire un enchaînement \lstinline|if| imbriqués pour contrôler l’accès :  
% badge OK, tenue OK ou formation OK, ou responsable.  
% Puis réécrire en une seule expression booléenne.
% \evaluationProfLignes{4}{4}
% \end{question}
% }

% \element{niveau3}{
% \begin{question}{n3-cond-court1}
% Que va afficher ?
% \begin{lstlisting}[language=C]
% int j=0, i=5;
% if( (j!=0) && (i/j>2) ) puts("OK"); else puts("SAFE");
% \end{lstlisting}
% \begin{reponses}
%   \bonne{SAFE}
%   \mauvaise{OK}
%   \mauvaise{Erreur}
%   \mauvaise{Rien}
%   \mauvaise{Toujours vrai}
%   \mauvaise{Toujours faux}
% \end{reponses}
% \end{question}
% }

% \element{niveau3}{
% \begin{question}{n3-cond-court2}
% Pourquoi \lstinline|(y!=0)&&(x%y==1)| est-il plus sûr que l’inverse ?
% \evaluationProfLignes{2}{2}
% \end{question}
% }

% \element{niveau3}{
% \begin{question}{n3-open-reduction}
% Rédiger une condition : réduction si étudiant ET (bourse OU mention).  
% Proposer deux cas de test.
% \evaluationProfLignes{3}{3}
% \end{question}
% }

% \element{niveau3}{
% \begin{question}{n3-cond-equivalence}
% Simplifier \lstinline|!(x<5)&&!(x>10)|.
% \begin{reponses}
%   \bonne{x>=5 && x<=10}
%   \mauvaise{x<5 || x>10}
%   \mauvaise{x<=5 && x>=10}
%   \mauvaise{Toujours vrai}
%   \mauvaise{Toujours faux}
%   \mauvaise{Erreur}
% \end{reponses}
% \end{question}
% }

% \element{niveau3}{
% \begin{question}{n3-open-bool}
% Écrire une condition : alarme si (temp>Tmax ET !override) OU (fumée ET ventilOff).  
% Donner un exemple vrai et un exemple faux.
% \evaluationProfLignes{3}{3}
% \end{question}
% }

% \element{niveau3}{
% \begin{question}{n3-cond-ouexcl}
% Quelle expression teste le OU exclusif (XOR) de p et q ?
% \begin{reponses}
%   \bonne{(p||q)&&!(p&&q)}
%   \mauvaise{p||q}
%   \mauvaise{p&&q}
%   \mauvaise{!(p||q)}
%   \mauvaise{!(p)&&!(q)}
%   \mauvaise{Erreur}
% \end{reponses}
% \end{question}
% }

% \element{niveau3}{
% \begin{question}{n3-open-exemples}
% Donner trois exemples de conditions imbriquées rencontrées dans des applications réelles.
% \evaluationProfLignes{3}{3}
% \end{question}
% }

% \element{niveau3}{
% \begin{question}{n3-cond-syntaxe}
% Laquelle est correcte ?
% \begin{reponses}
%   \bonne{\lstinline[language=C]|if((x>0 && y>0) || isAdmin) printf("OK");|}
%   \mauvaise{\lstinline[language=C]|if(x>0 and y>0 or isAdmin)|}
%   \mauvaise{\lstinline[language=C]|if((x>0 && y>0) || isAdmin) { printf("OK") }|}
%   \mauvaise{\lstinline[language=C]|if x>0 && y>0 printf("OK");|}
%   \mauvaise{\lstinline[language=C]|if((x>0)&&(y>0)||(isAdmin)); printf("OK");|}
%   \mauvaise{Aucune}
% \end{reponses}
% \end{question}
% }
