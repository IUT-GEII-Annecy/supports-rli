% ==================== NIVEAU 1 — VARIABLES / TYPES (10) ====================

\element{niveau1}{
\begin{question}{n1-types-annee-naissance}
Quel type est cohérent pour représenter l'année de naissance d'un étudiant ?
\begin{multicols}{5}
\begin{reponses}
  \bonne{\lstinline[language=c]|int|}
  \mauvaise{\lstinline[language=c]|float|}
  \mauvaise{\lstinline[language=c]|double|}
  \mauvaise{\lstinline[language=c]|string|}
  \mauvaise{\lstinline[language=c]|bool|}
\end{reponses}
\end{multicols}
\end{question}
}

\element{niveau1}{
\begin{question}{n1-types-initiale}
Quel type convient pour stocker une \textbf{initiale} (une seule lettre) ?
\begin{multicols}{5}
\begin{reponses}
  \bonne{\lstinline[language=c]|char|}
  \mauvaise{\lstinline[language=c]|int|}
  \mauvaise{\lstinline[language=c]|float|}
  \mauvaise{\lstinline[language=c]|double|}
  \mauvaise{\lstinline[language=c]|string|}
\end{reponses}
\end{multicols}
\end{question}
}

\element{niveau1}{
\begin{question}{n1-types-temperature}
Pour une température mesurée avec décimales, le type le plus adapté est :
\begin{reponses}
  \bonne{\lstinline[language=c]|float|}
  \mauvaise{\lstinline[language=c]|int|}
  \mauvaise{\lstinline[language=c]|char|}
  \mauvaise{\lstinline[language=c]|bool|}
\end{reponses}
\end{question}
}

\element{niveau1}{
\begin{question}{n1-types-salle}
Le \textbf{numéro de salle} (par ex. 105) devrait être stocké en :
\begin{reponses}
  \bonne{\lstinline[language=c]|int|}
  \mauvaise{\lstinline[language=c]|float|}
  \mauvaise{\lstinline[language=c]|char|}
  \mauvaise{\lstinline[language=c]|double|}
\end{reponses}
\end{question}
}

\element{niveau1}{
\begin{question}{n1-types-prix}
Un \textbf{prix} (ex. 3.20€) doit être stocké idéalement en :
\begin{reponses}
  \bonne{\lstinline[language=c]|double|}
  \mauvaise{\lstinline[language=c]|int|}
  \mauvaise{\lstinline[language=c]|char|}
  \mauvaise{\lstinline[language=c]|bool|}
  \mauvaise{\lstinline[language=c]|float|}
\end{reponses}
\end{question}
}

\element{niveau1}{
\begin{question}{n1-open-declarations}
Écrire la \textbf{déclaration} (sans initialisation) de variables pour : prénom, âge, taille en cm.
\evaluationProfLignes{3}{3}
\end{question}
}


\element{niveau1}{
\begin{question}{n1-trace-affectations}
Que valent \lstinline|x|, \lstinline|y|, \lstinline|z| après ce code ?
\begin{lstlisting}[language=C]
int x=2, y=5, z;
z = x + y;
y = z - 1;
x = y + z + x;
\end{lstlisting}
\begin{reponses}
  \bonne{\lstinline|x=13, y=6, z=7|}
  \mauvaise{\lstinline|x=7, y=6, z=13|}
  \mauvaise{\lstinline|x=6, y=7, z=13|}
  \mauvaise{\lstinline|x=12, y=7, z=6|}
\end{reponses}
\end{question}
}

\element{niveau1}{
\begin{question}{n1-prediction-division}
Prédire l'affichage :
\begin{lstlisting}[language=C]
int a=7, b=2; float x=7, y=2;
int c = a/b;
float d = x/y;
printf("C:%d\n", c);
printf("D:%f\n", d);
\end{lstlisting}
\begin{reponses}
  \bonne{\lstinline|C:3| et \lstinline|D:3.500000|}
  \mauvaise{\lstinline|C:3| et \lstinline|D:3.000000|}
  \mauvaise{\lstinline|C:3.500000| et \lstinline|D:3|}
  \mauvaise{\lstinline|C:3| et \lstinline|D:3.5|}
\end{reponses}
\end{question}
}

% ==================== NIVEAU 1 — CONDITIONS (10) ====================

\element{niveau1}{
\begin{question}{n1-if-majeur}
Que doit afficher ce code si \lstinline|age=16| ?
\begin{lstlisting}[language=C]
if (age >= 18) printf("Bienvenue\n");
else printf("Au revoir\n");
\end{lstlisting}
\begin{reponses}
  \bonne{\lstinline|Au revoir|}
  \mauvaise{\lstinline|Bienvenue|}
  \mauvaise{Rien}
  \mauvaise{Erreur}
\end{reponses}
\end{question}
}

% \element{niveau3}{
% \begin{question}{n1-parite}
% Quelle condition teste que \lstinline|n| est \textbf{pair} ?
% \begin{reponses}
%   \bonne{\lstinline[language=C]|n % 2 == 0|}
%   \mauvaise{\lstinline[language=C]|n / 2 == 0|}
%   \mauvaise{\lstinline[language=C]|n == 2|}
%   \mauvaise{\lstinline[language=C]|n % 2 == 1|}
% \end{reponses}
% \end{question}
% }

\element{niveau2}{
\begin{question}{n1-intervalle}
Condition correcte pour \(5 \le x \le 10\) (inclus) :
\begin{reponses}
  \bonne{\lstinline[language=C]|x >= 5 && x <= 10|}
  \mauvaise{\lstinline[language=C]|x > 5 || x < 10|}
  \mauvaise{\lstinline[language=C]|x > 5 && x < 10|}
  \mauvaise{\lstinline[language=C]|x <= 5 && x >= 10|}
\end{reponses}
\end{question}
}

\element{niveau1}{
\begin{question}{n1-open-comparaison}
Écrire une suite \lstinline|if/else if/else| qui affiche le plus grand entre \lstinline|a| et \lstinline|b|, ou \og Égaux \fg{}.
\evaluationProfLignes{3}{3}
\end{question}
}

\element{niveau1}{
\begin{question}{n1-prediction-simple}
Sortie du programme ?
\begin{lstlisting}[language=C]
int t=5;
if (t>5) printf(">5\n"); else printf("<=5\n");
\end{lstlisting}
\begin{reponses}
  \bonne{\lstinline|<=5|}
  \mauvaise{\lstinline|>5|}
  \mauvaise{Rien}
  \mauvaise{Erreur}
\end{reponses}
\end{question}
}

\element{niveau1}{
\begin{question}{n1-open-nbmystere}
On a \lstinline|int mystere=7;| et l'utilisateur propose \lstinline|p|.  
Décrire la logique pour afficher \og Bravo \fg{} si \lstinline|p==mystere|, sinon \og Raté \fg{}.  
Proposer une amélioration : \og Trop petit \fg{} / \og Trop grand \fg{}.
\evaluationProfLignes{3}{3}
\end{question}
}

\element{niveau1}{
\begin{question}{n1-if-syntaxe}
Quelle écriture est \textbf{correcte} ?
\begin{reponses}
  \bonne{\lstinline[language=C]|if (age>=18) { printf("Majeur"); } else { printf("Mineur"); }|}
  \mauvaise{\lstinline[language=C]|if (age>=18) printf("Majeur") else printf("Mineur");|}
  \mauvaise{\lstinline[language=C]|if age>=18 then printf("Majeur");|}
  \mauvaise{\lstinline[language=C]|if (age>=18) { printf("Majeur") } else printf("Mineur")|}
\end{reponses}
\end{question}
}

\element{niveau1}{
\begin{question}{n1-prediction-multi}
Sortie du programme ?
\begin{lstlisting}[language=C]
int a=3, b=3;
if (a>b) puts("A");
else if (a<b) puts("B");
else puts("E");
\end{lstlisting}
\begin{reponses}
  \bonne{\lstinline|E|}
  \mauvaise{\lstinline|A|}
  \mauvaise{\lstinline|B|}
  \mauvaise{Rien}
\end{reponses}
\end{question}
}

\element{niveau1}{
\begin{question}{n1-open-cond-rediger}
Rédiger une condition qui affiche \og Accès \fg{} si \lstinline|badge==1|, sinon \og Refus \fg{}.  
\evaluationProfLignes{3}{1}
\end{question}
}
