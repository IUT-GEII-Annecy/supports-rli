
\section{Hello, World : Premier programme}

\begin{lstlisting}[language=c]
#include <stdio.h>

int main(void)
{
    printf("hello, world\n");
}
\end{lstlisting}

\subsection{Explications}
Le programme précédent peut se décomposer de la façon suivante :

\begin{itemize}
	\item Import de la bibliothèque pour afficher du texte.
	      \begin{lstlisting}[language=c]
#include <stdio.h>
\end{lstlisting}
\end{itemize}

\begin{itemize}
	\item Fonction principale :
	      \begin{lstlisting}[language=c]
int main(void)
\end{lstlisting}
	      \begin{itemize}
		      \item Le contenu de cette fonction est compris entre les accolades : \texttt{\{ ... \}}
	      \end{itemize}
\end{itemize}

\begin{itemize}
	\item Appel de la fonction \texttt{printf}
	      \begin{lstlisting}[language=c]
printf("...");  
\end{lstlisting}
	\item \texttt{\textbackslash n} demande un retour à la ligne.
	\item En C, chaque instruction se termine par un \texttt{;}
\end{itemize}

\subsection{Compilation et exécution}
\texttt{Encore une histoire de 1 et de 0 !}
Pour le moment, notre programme est écrit sous la forme de texte. Il est compréhensible par les humains (développeurs). Il faut, à présent, le convertir en un langage que le processeur comprend : des instructions binaires.

C'est le rôle du \texttt{compilateur} : Créer un fichier exécutable à partir d'un code \texttt{.c}. Dans ce module, cela se fera avec la commande \texttt{make}. Ensuite, on peut exécuter le programme.
\begin{lstlisting}[language=bash,style=console]
make hello
./hello
\end{lstlisting}

\begin{UPSTIinfor}{Étapes}
	\begin{enumerate}
		\item \texttt{make hello} → compile.
		\item \texttt{./hello} → exécute.
	\end{enumerate}
\end{UPSTIinfor}

Résultat :
\begin{lstlisting}[language=bash,style=console]
hello, world
\end{lstlisting}

\begin{UPSTIwarning}{Erreurs fréquentes}

	\begin{itemize}
		\item Oublier le \texttt{;} → erreur de compilation.
		\item Oublier \texttt{\#include <stdio.h>} → \texttt{printf} inconnu.
		\item Écrire le texte à afficher sans guillemets → interprété comme variable.
	\end{itemize}
\end{UPSTIwarning}