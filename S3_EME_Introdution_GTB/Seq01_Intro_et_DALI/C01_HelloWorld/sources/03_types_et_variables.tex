\section{Types et Variables}
\begin{UPSTIinfor}{Encore une histoire de binaire}
	On l'a vu, toutes les données sont représentées sous forme binaire (des 1 et des 0). C'est aussi le cas pour les variables que l'on utilisera.
	Or, l'ordinateur doit pouvoir déterminer s'il faut interpréter le mot binaire de cette variable comme un nombre, une lettre, une couleur ou autre.
	Il donc faut préciser à quel type est associé chacune des variables que l'on déclare.
\end{UPSTIinfor}

\subsection{Les types de données en C}

Chaque variable a un type précis :
\begin{center}
	\begin{tabular}{|l|l|l|}
		\hline
		Type            & Description                             & Exemple                    \\
		\hline
		\texttt{int}    & nombres entiers                         & \texttt{10}, \texttt{-5}   \\
		\texttt{char}   & un seul caractère                       & \texttt{'a'}, \texttt{'Z'} \\
		\texttt{float}  & nombres décimaux approximatifs          & \texttt{3.14}              \\
		\texttt{double} & nombres décimaux avec plus de précision & \texttt{3.141592}          \\
		\texttt{long}   & entiers plus grands que \texttt{int}    & \texttt{123456789}         \\
		\texttt{string} & texte (via la bibliothèque CS50)        & \texttt{"Hello"}           \\
		\texttt{bool}   & vrai/faux (via la bibliothèque CS50)    & \texttt{true}              \\
		\hline
	\end{tabular}
\end{center}
\subsection{Déclaration et utilisation}

\begin{UPSTIinfor}{Déclaration de variable}
	Syntaxe : \texttt{<type> <nom>;}
	Exemple :
	\begin{lstlisting}[language=c]
int nombre;
\end{lstlisting}
	On peut affecter une valeur en même temps :
	\begin{lstlisting}[language=c]
int nombre = 42;
\end{lstlisting}
\end{UPSTIinfor}

\subsection{Afficher une variable dans un texte}
Pour afficher une variable dans une phrase, on utilise \texttt{printf} avec un \textbf{placeholder}.

\begin{lstlisting}[language=c]
#include <stdio.h>

int main(void)
{
    int age = 20;
    printf("J'ai %i ans.\n", age);
}
\end{lstlisting}


\subsubsection{Placeholders courants}

\begin{center}
	\begin{tabular}{|l|l|l|l|}
		\hline
		Spécificateur  & Description                    & Exemple d'utilisation                                    & Sortie                 \\
		\hline
		\texttt{\%i}   & entier                         & \texttt{printf("Valeur : \%i\textbackslash n", 42);}     & \texttt{Valeur : 42}   \\
		\texttt{\%f}   & flottant (nombre à virgule)    & \texttt{printf("Pi = \%f\textbackslash n", 3.141593);}   & \texttt{Pi = 3.141593} \\
		\texttt{\%.2f} & flottant arrondi à 2 décimales & \texttt{printf("Pi = \%.2f\textbackslash n", 3.141593);} & \texttt{Pi = 3.14}     \\
		\texttt{\%c}   & caractère                      & \texttt{printf("Lettre : \%c\textbackslash n", 'A');}    & \texttt{Lettre : A}    \\
		\texttt{\%s}   & texte (string)                 & \texttt{printf("Bonjour \%s\textbackslash n", "Alice");} & \texttt{Bonjour Alice} \\
		\hline
	\end{tabular}
\end{center}
\begin{UPSTIinfor}{Exemple}
	\begin{lstlisting}[language=c]
float pi = 3.14159;
printf("Pi = %.2f\n", pi);
\end{lstlisting}
	Affiche : \texttt{Pi = 3.14}
\end{UPSTIinfor}


\subsubsection{Imprécision des flottants}
Puisqu'ils sont codés sur un nombre de bits finis, les nombres à virgule ont une précision limitée :

\begin{lstlisting}[language=c]
float x = 1.0 / 10.0;
printf("%.20f\n", x);
\end{lstlisting}

Affiche par exemple :
\begin{lstlisting}[language=bash,style=console]
0.10000000149011611938
\end{lstlisting}

\begin{UPSTIinfor}{Conseil}
	Toujours limiter l’affichage des décimales avec \texttt{\%.2f}, \texttt{\%.3f}, etc.
\end{UPSTIinfor}