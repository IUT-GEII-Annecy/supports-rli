\resetNumActivite{}
\section{Travail à réaliser}
\UPSTIinfo[Variables d'entrées utiles et adresses DALI]{
    \label{info:var}
    \begin{minipage}{.5\textwidth}
    \begin{tabular}{|c|c|}
        \hline
        \textbf{Element} & \textbf{Nom de variable}\\\hline
        Bouton arret urgence & ixArretUrgence\\
        Bouton poussoir Blanc & ixBpAcq \\\hline
        Bouton poussoir Rouge & ixBpRouge \\
        Bouton poussoir Vert & ixBpVert \\
        Bouton poussoir Bleu & ixBpBleu \\\hline
        Sélecteur Haut & ixSelUp \\
        Sélecteur Bas & ixSelDown \\\hline
    \end{tabular}
\end{minipage}\hfill
\begin{minipage}{.5\textwidth}
    \begin{flushright}
    \begin{tabular}{|l|c|}
        \hline
        \textbf{Element DALI} & \textbf{Adresse DALI}\\\hline
        \multirow{4}{4cm}{Luminaires (de la fenêtre vers le couloir)} & 1 \\
        & 2 \\
        & 3 \\ 
        & 4 \\\hline
        {Lampe Rouge} & 10 \\
        {Lampe Verte} & 11 \\
        {Lampe Bleue} & 12 \\
        \hline
    \end{tabular}
\end{flushright}
\end{minipage}
}
\subsection{Observation matérielle}
\begin{UPSTIactivite}[][Elements du système]
    \label{act:tpElts}
    \UPSTIquestion{Noter la référence de votre automate et repérer le coupleur DALI (750-641) sur votre maquette.}
\end{UPSTIactivite}

\subsection{Configuration du projet}
\begin{UPSTIactivite}[][Préparation du projet]
    \UPSTIetape{Lancer la machine virtuelle \textit{ARS (Win10)}}
    \UPSTIetape{Pendant qu'elle démarre : Sous votre répertoire associé à ce module, copier le squelette de l'application \textbf{correspondant à l'automate de votre plateforme} qui se trouve sous \\ \textbf{U:\textbackslash Documents\textbackslash BUT\textbackslash GEII\textbackslash ModuleS3\textbackslash R3.13 Réseaux Spécialisés EME \textbackslash Squelettes \textbackslash Dali}. }
    \UPSTIetape{Lancer le logiciel \textbf{CoDeSys V2.3} sur la machine virtuelle et ouvrir le squelette de l'application.}

    Nous allons à présent inclure la bibliothèque DALI\_02.lib dans notre projet afin de pouvoir utiliser les blocs fonctionnels nécessaires à la communication avec le bus DALI.
    
    \UPSTIetape{Dans l'onglet \textbf{Ressources}, ouvrir le \textbf{Gestionnaire de bibliothèques}. Faire un clic droit pour ajouter une bibliothèque et sélectionner \textbf{DALI\_02.lib}. }
    Elle se trouve dans le répertoire \textbf{C:\textbackslash Program Files\textbackslash WAGO Software\textbackslash CoDeSys V2.3\textbackslash Targets\textbackslash WAGO\textbackslash Librairies\textbackslash Building}

    \begin{center}\includegraphics[width=.3\textwidth]{insererBibliotheque} \hspace{1cm}\includegraphics[width=.3\textwidth]{ajoutBibliothequeDALI}
    \end{center}
\end{UPSTIactivite}


\subsection{Premières commandes DALI}
Dans cette section, nous allons mettre en place notre première communication DALI. Nous allons commander un plafonnier à partir d'un bouton poussoir en utilisant un bloc fonction de la bibliothèque DALI\_02.lib.

\begin{UPSTIactivite}[][Programme en CFC avec blocs fonction]
    \UPSTIetape{Ajouter un nouveau module :
    \begin{itemize}
        \item Dans l'onglet \textbf{Modules}, ajouter un nouveau module (\textit{Clique-droit -> Insérer objet})
        \item Le nommer \textit{Plafonniers} en langage CFC. L'appeler dans le programme DALI
        \end{itemize}}
    \UPSTIetape{Dans ce nouveau programme en CFC :
        \begin{itemize}
            \item Déclarer un bloc fonction \textbf{fbJobList} de type \textbf{FbDALI\_Joblist}
            \item Lui donner la valeur par défaut \textbf{bModule\_750\_641 := 1} (Numéro du module DALI)
            \item Placer ce bloc dans le programme CFC (Tutoriel~\ref{tuto:insertionBloc})
        \end{itemize}}
    \UPSTIetape{Déclarer et insérer un bloc fonction \textbf{fbTelerupteur} de type \textbf{FbDALI\_LatchingRelay}. \textit{(Prépa \ref{prepa:telerupteur}, Tuto~\ref{tuto:insertionBloc})}}
    \UPSTIetape{Connecter les entrées et sorties des blocs en CFC pour réaliser le cahier des charges du télérupteur.}
    \UPSTIetape{Faire vérifier et tester votre programme.}
\end{UPSTIactivite}

\UPSTItuto[Déclaration de blocs fonction dans CoDeSys]{%
\label{tuto:declarationAutomatique}
    Dans le logiciel CoDeSys, pour déclarer un bloc fonction (par exemple \textbf{FbDALI\_Joblist}) :
    \begin{enumerate}
        \item Cliquer dans la zone de déclaration des variables
        \item Clique-droit sur une ligne vide
        \begin{itemize}
            \item \textit{Déclarer automatiquement}
        \end{itemize}
        \item Cliquer sur les $\dots$ de \textbf{Type} :
        \begin{itemize}
            \item Type défini
            \item Sélectionner le type de bloc fonction dans la bibliothèque DALI
        \end{itemize}
        \item Cliquer sur les $\dots$ de \textbf{Valeurs initiales} (si nécessaire) :
        \begin{itemize}
            \item Définir les valeurs par défaut des paramètres du bloc
        \end{itemize}
        \item Donner un nom à la variable dans le champ \textbf{Nom} (exemple : \textbf{fbJobList})
    \end{enumerate}
    \begin{center}
        \includegraphics[width=.75\textwidth]{declarationAutomatique_menu}
    \end{center}
}%

\UPSTItuto[Insertion d'un bloc fonction]{
    \label{tuto:insertionBloc}
    Pour insérer un bloc fonction dans un programme CFC : 
    \begin{enumerate}
        \item Cliquer sur l'icone Module \includegraphics[height=10pt]{tuto/iconeModule}
        \item Descendre la souris dans la zone de bloc
        \item Renommer le bloc \textit{AND} par le type de bloc que vous voulez insérer (\textit{fbDALI\_Joblist}, par exemple)
        \\\begin{center}
            \includegraphics[clip, trim= 1mm 5mm 1mm 1mm,height=1.2cm]{tuto/blocVierge}
        \end{center}
        A cette étape, le bloc est inséré, mais n'est pas encore appelé. Il faut lui associer une variable déclarée. 
        \item Double-cliquer sur les \textcolor{red}{\textbf{???}} et lui associer une variable existante. 
        \\\begin{center}
            \includegraphics[clip, trim= 1mm 3mm 1mm 1mm,height=1.2cm]{tuto/blocOk}
        \end{center}
        \item Ajouter une entrée (\includegraphics[height=10pt]{tuto/iconeEntree}) et une sortie (\includegraphics[height=10pt]{tuto/iconeSortie}) puis leur associer une variable ou une valeur. 
        \\\begin{center}
            \includegraphics[clip, trim= 1mm 3mm 1mm 3mm,height=1.2cm]{tuto/blocComplet}
        \end{center}
    \end{enumerate}
}
\vspace{-.8cm}
\begin{UPSTIactivite}[][Appel d'un bloc en ST]
    Dans cette activité, nous allons réaliser la même chose mais en appelant les blocs fonction en langage ST.
    \UPSTIetape{Dans le programme DALI (en ST) :}
    \begin{itemize}
        \item Déclarer une variable \textbf{fbLatchingRelay2} de type \textbf{FbDALI\_LatchingRelay} \textit{(Tutoriel \ref{tuto:declarationAutomatique})}
        \item Appeler le bloc fonction en ST (Tutoriel \ref{tuto:aideSaisie}) et lui associer les paramètres d'entrées et sorties nécessaires
        \item Faire vérifier et tester votre programme
    \end{itemize}
\end{UPSTIactivite}


\vspace{-1cm}

\UPSTItuto[Aide à l'appel d'élément déclaré]{
    \label{tuto:aideSaisie}
    Dans le logiciel CoDeSys, il est possible d'utiliser une variable définie ou d'appeler une fonction ou un bloc fonction déclaré facilement en suivant les étapes suivantes :
    \begin{itemize}
        \item Clique droit sur une ligne vide
        \begin{itemize}
            \item \textit{Liste de sélection pour l'édition}
        \end{itemize} 
        \item Chercher l'élément que vous souhaitez appeler 
    \end{itemize}
    \begin{center}
        \includegraphics[width=.7\textwidth, clip,trim=0mm 55mm 0mm 0mm]{aideSaisie_menu}
    \end{center}
}


\subsection{Envoi de commandes plus complexes}

\begin{UPSTIactivite}[][Mise en place de la communication avec le ballast Rouge en CFC]
    \UPSTIetape{Créer un nouveau programme en CFC nommé \textbf{LampeRouge}}
    \UPSTIetape{Déclarer les variables nécessaires :
    \begin{itemize}
        \item \textbf{fbMasterRouge} : bloc fonction de type \textbf{FbDALI\_Master}
        \item \textbf{bAdresseRouge} : BYTE := 10 (adresse DALI de la lampe rouge)
        \item \textbf{bValeurCommande} : BYTE (valeur de luminosité)
        \item \textbf{xDemarrer} : BOOL (signal de déclenchement)
    \end{itemize}}
    \UPSTIetape{En CFC, placer le bloc \textbf{fbMasterRouge} et connecter directement :
    \begin{itemize}
        \item L'adresse DALI (entrée bAdresseRouge)
        \item La valeur de commande (entrée bValeurCommande)
        \item Le signal de démarrage (entrée xDemarrer)
    \end{itemize}}
\end{UPSTIactivite}

\UPSTIremarque[Qu'avons-nous fait jusque là ?]{
    Les étapes précédentes ont permis de déclarer les variables et les blocs fonctionnels nécessaires à la mise en place d'un bus DALI.

    Le bloc fonction \textbf{FbDALI\_Master} scrute sa variable d'entrée \textbf{xStartDaliMaster} dans l'attente d'une mise à 1. Lorsque cela se produit, il exécute la commande DALI paramétrée sur ses autres entrées (adresse, valeur de commande, etc.).

    Le bloc fonction \textbf{FbDALI\_Joblist} se charge d'envoyer les commandes que les blocs \textbf{FbDALI\_Master} auront générées vers le bus DALI physique.

    Dans ce TP, nous utilisons des variables simples (BYTE, BOOL) pour connecter directement les entrées des blocs fonction, ce qui facilite la compréhension du flux de données.
}

\begin{UPSTIactivite}[][Commande directe]
    \UPSTIetape{Écrire un programme qui allume la lampe rouge à 75\% de sa luminosité maximale lorsque le bouton rouge est appuyé et qui l'éteint lorsque le bouton est relâché.} Faire vérifier par l'enseignant.
\end{UPSTIactivite}

\UPSTIattention{Si le programme précédent est fait trop simplement, des commandes DALI sont envoyées sans arrêt, saturant le réseau DALI. Il faut corriger cela en utilisant, par exemple, la détection de front montant.}

\begin{UPSTIactivite}[][Correction du programme précédent]
    \UPSTIetape{Si nécessaire, corriger le programme précédent pour n'envoyer qu'une seule commande DALI à chaque changement d'état.}
\end{UPSTIactivite}

\pagebreak
\section{Programmes avancés}

\UPSTIboiteCentrale{Un bouton par couleur}{
    \begin{itemize}
        \item Le bouton poussoir rouge commande la lampe rouge.
        \item Le bouton poussoir vert commande la lampe verte.
        \item Le bouton poussoir bleu commande la lampe bleue.
    \end{itemize}
}

\UPSTIboiteCentrale{Cahier des charges : Intensité lumineuse}{
    \begin{itemize}
        \item Le sélecteur 3 position modifie l'intensité de la lampe rouge
    \end{itemize}}


\begin{UPSTIactivite}
    \UPSTIetape{Programmer le cahier des charges \textit{Un bouton par couleur}. Tester le programme}
    \UPSTIetape{Programmer le cahier des charges \textit{Intensité lumineuse}. Tester le programme}
\end{UPSTIactivite}

\UPSTIboiteCentrale{Cahier des charges : Sélection de couleur}{
    \begin{itemize}
        \item Chaque bouton de couleur permet de sélectionner la couleur de la lampe à commander.
        \item Le sélecteur à 3 boutons permet d'augmenter ou de diminuer l'intensité lumineuse de la lampe sélectionnée.
    \end{itemize}}

\begin{UPSTIactivite}
    Pour coder le cahier des charges ci-dessus, on propose la stratégie suivante :
    \UPSTIetape{Créer trois variables pour gérer les trois lampes :
    \begin{itemize}
        \item \textbf{bCouleurSelectionnee} : BYTE (0=Rouge, 1=Vert, 2=Bleu)
        \item \textbf{abAdressesLampes} : ARRAY[0..2] OF BYTE := [10, 11, 12]
        \item \textbf{abIntensite} : ARRAY[0..2] OF BYTE (intensité de chaque lampe)
    \end{itemize}}
    \UPSTIetape{Utiliser les boutons de couleur pour sélectionner la lampe à commander (modifier \textbf{bCouleurSelectionnee})}
    \UPSTIetape{Utiliser le sélecteur pour modifier l'intensité de la lampe sélectionnée}
    \UPSTIetape{En ST, envoyer les commandes aux lampes en utilisant \textbf{FbDALI\_Master}}
    \UPSTIetape{Tester et faire vérifier votre programme}
\end{UPSTIactivite}

\begin{UPSTIactivite}[][Appeler l'enseignant]
    \vspace{4cm}
\end{UPSTIactivite}

