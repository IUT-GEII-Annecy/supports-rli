\begin{UPSTIactivite}[2][Le protocole RS 485][][][Préparation]
	\label{prepa:rs485}%
	\UPSTIquestion{A partir du TD 2, rappeler le fonctionnement du protocole RS 485.}
    \begin{itemize} 
        \item Quelles tensions sont utilisées ? 
        \item Quelle est la vitesse de transmission ?
        \item Quelle est la longueur maximale du bus ?
        \item Quelle est la différence avec le protocole RS 232 ?
        \item Expliquer la notion de bit de parité et de bit de stop.
        \end{itemize}
\end{UPSTIactivite}

Une trame DMX512 contient dans sa partie utile 512 octets, qui par leur position dans la trame définit la valeur de
commande des 512 canaux. La norme laisse le soin à chaque constructeur de choisir l’interprétation de ces valeurs
(comprises entre 0 et 255).
\begin{UPSTIactivite}[2][Vitesse du DMX][][][Préparation]
     \UPSTIquestion{Combien de trame par seconde peut-on envoyer via le réseau DMX ?}
\end{UPSTIactivite}

\subsection{Architecture du TP}
Dans ce TP, on se propose d'utiliser un coupleur \textbf{750-652} pour mettre en oeuvre la commande d'un module DMX. Il jouera donc le rôle de maître DMX. 

On reliera deux ballasts RGBW (Red Green Blue White). Sur ces deux ballasts, on fixe le canal de référence à l'aide de boutons poussoirs \textit{CH+} et \textit{CH-}. Ce canal de référence correspond à la sortie \textbf{R}, puis les canaux suivants seront respectivement \textbf{G}, \textbf{B} puis \textbf{W}. 
Les éléments DMX de notre maquette sont installés en \textit{Daisy Chain}. Cela signifie que la sortie DMX du maître est reliée à l'entrée DMX du premier esclave, la sortie DMX de ce premier esclave est reliée à l'entrée DMX du second esclave, et ainsi de suite. La dernière sortie DMX de la chaîne n'est pas reliée.

\begin{UPSTIactivite}[2][Architecture matérielle][][][Préparation]
    \UPSTIquestion{Expliquer ce que fait le module \textbf{750-652}.}\\
    \UPSTIquestion{Proposer un schéma structurel de l'architecture matérielle mise en place pour ce TP.}
    \UPSTIquestion{Choisir des numéros de canaux pour chaque ballast RGBW et les indiquer sur le schéma structurel.} 
\end{UPSTIactivite}

\subsection{Programmation}

\begin{UPSTIactivite}[2][DMX Master][][][Préparation]
    \UPSTIquestion{A l'aide de l'annexe page \pageref{appendix:DMXMaster}, expliquer le rôle du bloc \textit{FbDMX\_Master}.}
    \UPSTIquestion{De quel type et à quoi sert la variable \textbf{abDMX\_Values}?}
\end{UPSTIactivite}
\UPSTIboiteCentrale{Cahier des charges : Une couleur}{
    On souhaite allumer la colonne de gauche d'une en orange \#ff4524. Ce code, au format HTML, donne l'intensité des trois canaux Rouge, Vert et Bleu codé en hexadécimal (R = ff, G = 45, B = 24).
}
\begin{UPSTIactivite}[2][Une couleur][][][Préparation]
    \UPSTIquestion{Convertir les codes hexadécimaux des trois couleurs au format décimal et BYTE.}
    \UPSTIquestion{Écrire un programme, en langage CFC (Blocs fonctions), permettant d'allumer la colonne de gauche de la couleur choisie.}
    \UPSTIquestion{Ecrire le même programme en langage ST (Structured Text).}
\end{UPSTIactivite}

\UPSTIboiteCentrale{Cahier des charges : Succession de couleurs}{
    \label{CDC:succession}
    On souhaite effectuer une séquence de 10 couleurs différentes sur la colonne de droite (Canal 7).
    La séquence de couleur est définie dans le tableau suivant : 

    \begin{tabular}{|ccccccccccc|}
        \hline
        Couleur & 1 & 2 & 3 & 4 & 5 & 6 & 7 & 8 & 9 & 10 \\\hline
        \hline
    B & 16\#00 & 16\#00 & 16\#00 & 16\#00 & 16\#00 & 16\#08 & 16\#10 & 16\#20 & 16\#40 & 16\#80 \\\hline
    G & 16\#08 & 16\#10 & 16\#20 & 16\#40 & 16\#80 & 16\#80 & 16\#40 & 16\#20 & 16\#10 & 16\#08 \\\hline
    R & 16\#80 & 16\#40 & 16\#20 & 16\#10 & 16\#08 & 16\#00 & 16\#00 & 16\#00 & 16\#00 & 16\#00 \\\hline
    \end{tabular}
}



\begin{UPSTIactivite}[2][Succession de couleurs][][][Préparation]
    Pour ce cahier des charges, on propose d'utiliser le bloc fonction \textit{FbRGB\_RecallColourPalette} (page~\pageref{appendix:RGBCross}) qui permet de faire varier la couleur d'un canal de manière progressive.\\
    \UPSTIquestion{Déclarer un tableau permettant de stocker 10 couleurs.}
    \UPSTIquestion{Écrire un programme, en langage ST, permettant de faire varier la couleur de la colonne à l'aide d'un potentiomètre donnant sa valeur sur une variable \textit{bValue} de type \textbf{BYTE}.}
    On souhaite que les intervalles de couleurs soient les plus réguliers possible en fonction de la valeur du potentiomètre.
\end{UPSTIactivite}
