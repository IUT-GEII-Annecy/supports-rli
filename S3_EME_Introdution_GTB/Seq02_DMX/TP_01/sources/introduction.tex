Dans les bâtiments tertiaires et secondaires, 
les techniques modernes de commande d’éclairage sont réalisées via le réseau non propriétaire DALI.
Néanmoins, la dynamique de ce réseau reste trop lente par rapport aux exigences
des effets de lumière que l’on souhaite faire sur les scènes de spectacle. Ainsi,
dans le cadre de la commande de l’appareillage scénique (projecteurs RVB, lyres,
stroboscopes, machines à fumée, gobo, \dots), les standards utilisés pour piloter ces
équipements via des réseaux sont : pour le plus répandu DMX512 (cf. United States
Institute for Theatre Technology : DMX512-A - Asynchronous Serial Digital Data
Transmission), pour le plus récent Art-Net (cf. http://art-net.org.uk) ou bien
encore les standards MIDI, Cobranet, \dots

\UPSTIobjectif{Mettre en oeuvre le standard \textbf{DMX 512}, premier réseau à s’être imposé dans le monde du
spectacle, à l’aide d’une structure de commande basée sur un automate programmable industriel.}

\section{Le réseau DMX}

DMX est l’acronyme de Digital MultipleX qui est la définition d’une interface
numérique standardisée non propriétaire pour les équipements scéniques. Le terme 512
fait référence au nombre de canaux disponibles sur un même réseau (appelé univers). La
norme prévoit d'utiliser au maximum 32 systèmes de 16 canaux chacun. Un équipement
DMX512 peut donc utiliser plusieurs canaux (par exemple, les ballasts RGBW que l’on
utilisera nécessitent quatre canaux consécutifs).

Comme le réseau DMX512 est unidirectionnel (du maitre vers les équipements uniquement), on ne dispose pas sur
ce réseau de compte rendu sur les échanges et sur la communication. Par conséquent, ce standard ne doit pas être
utilisé pour des spectacles ou des équipements réclamant de la sécurité (spectacles pyrotechniques, équipements
assurant le gréement de décors, leur levage, \dots).

\subsection{Caractéristiques techniques}
\begin{itemize}
    \item Tout équipement propose une entrée DMX IN(+, -), et une sortie DMX DMX IN
    DMX OUT
    OUT (+, -) (la sortie DMX OUT du dernier équipement sur la ligne doit
    être connectée à une résistance d’impédance \SI{120}{ohm} )
    \item Topologie de type Bus avec un câblage série de type « Daisy chain »
    \item Bus Maitre-Esclaves unidirectionnel
    Terminateur
    \item Transmission série : 8 bits, 2 bits de stop, sans parité (RS 485)
    de ligne
    \item Vitesse de transmission : 250000 bit/s
    R=\SI{120}{Ohm}
    \item Longueur des câbles : 300 mètres entre équipements
    \item Le pilotage simultané de plusieurs appareils s’effectue en leur attribuant
    les mêmes canaux
\end{itemize}



