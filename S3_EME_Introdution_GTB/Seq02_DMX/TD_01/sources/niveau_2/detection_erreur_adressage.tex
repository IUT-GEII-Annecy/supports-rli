\begin{UPSTIexercice}{Détection d'erreur d'adressage}
Un technicien a mal configuré l'adressage d'une installation comportant :
\begin{itemize}
    \item 1 projecteur RGB à l'adresse 10
    \item 1 stroboscope (1 canal) à l'adresse 11
    \item 1 autre projecteur RGB à l'adresse 13
\end{itemize}

\UPSTIquestion{Identifier le problème dans cette configuration.}
\UPSTIquestion{Expliquer les conséquences sur le fonctionnement des appareils.}
\UPSTIquestion{Proposer une correction de l'adressage pour que tous les appareils fonctionnent correctement.}
\end{UPSTIexercice}

\begin{UPSTIprofOnlyEnv}
    \begin{UPSTIcorrectionP}{Détection d'erreur d'adressage}
        \textbf{1)} Problème : Le premier projecteur RGB (adresse 10) utilise les canaux 10, 11 et 12. Le stroboscope est configuré sur le canal 11, qui est \textbf{déjà utilisé} par le projecteur RGB. Il y a un \textbf{conflit d'adressage}.

        \textbf{2)} Conséquences :
        \begin{itemize}
            \item Le stroboscope et le canal Vert du projecteur RGB écouteront tous deux le canal 11
            \item Le contrôle du projecteur RGB sera incorrect (impossible de régler le vert indépendamment)
            \item Le stroboscope s'activera quand on règle le vert du projecteur
        \end{itemize}

        \textbf{3)} Correction :
        \begin{center}
        \begin{tabular}{|c|c|c|}
        \hline
        \textbf{Appareil} & \textbf{Adresse DMX} & \textbf{Canaux utilisés} \\
        \hline
        Projecteur RGB 1 & 10 & 10, 11, 12 \\
        Stroboscope & 13 & 13 \\
        Projecteur RGB 2 & 14 & 14, 15, 16 \\
        \hline
        \end{tabular}
        \end{center}
    \end{UPSTIcorrectionP}
\end{UPSTIprofOnlyEnv}
