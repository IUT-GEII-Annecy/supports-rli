\begin{UPSTIexercice}{Brochage XLR}
On utilise des connecteurs XLR 5 broches pour une installation DMX professionnelle.

\UPSTIquestion{Donner la fonction de chacune des 5 broches du connecteur XLR 5.}
\UPSTIquestion{Quelles sont les deux broches essentielles pour la transmission du signal DMX ?}
\UPSTIquestion{Pourquoi le connecteur XLR 5 broches est-il préférable au XLR 3 broches pour le DMX ?}
\UPSTIquestion{Peut-on utiliser un câble XLR audio (micro) pour une installation DMX temporaire ? Justifier.}
\end{UPSTIexercice}

\begin{UPSTIprofOnlyEnv}
    \begin{UPSTIcorrectionP}{Brochage XLR}
        \textbf{1)} Brochage XLR 5 broches :
        \begin{itemize}
            \item Broche 1 : Masse / Blindage (GND)
            \item Broche 2 : Data- (DMX-)
            \item Broche 3 : Data+ (DMX+)
            \item Broche 4 : Réservé (seconde paire Data2-)
            \item Broche 5 : Réservé (seconde paire Data2+)
        \end{itemize}

        \textbf{2)} Broches essentielles : \textbf{2 (Data-)} et \textbf{3 (Data+)}

        \textbf{3)} Le XLR 5 broches permet :
        \begin{itemize}
            \item D'éviter la confusion avec les câbles audio XLR 3
            \item De prévoir une seconde paire pour des extensions futures (ex: RDM)
            \item Une meilleure séparation des signaux
        \end{itemize}

        \textbf{4)} En dépannage temporaire, oui, car les broches 1, 2, 3 sont câblées de la même façon. \textbf{MAIS} le câble audio n'a pas la bonne impédance (typiquement 75-110 $\Omega$ au lieu de 120 $\Omega$), ce qui peut causer des problèmes sur de longues distances ou avec beaucoup d'appareils. \textbf{Non recommandé pour une installation permanente}.
    \end{UPSTIcorrectionP}
\end{UPSTIprofOnlyEnv}
