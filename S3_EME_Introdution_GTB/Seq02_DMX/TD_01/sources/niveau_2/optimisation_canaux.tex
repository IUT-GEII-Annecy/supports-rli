\begin{UPSTIexercice}{Optimisation du nombre de canaux}
Dans la pratique, on ne transmet pas toujours les 512 canaux. Un contrôleur DMX peut optimiser la transmission en n'envoyant que les canaux réellement utilisés.

On considère une installation utilisant uniquement les canaux 1 à 24.

\UPSTIquestion{Calculer la durée d'une trame DMX optimisée transmettant uniquement 24 canaux.}
\UPSTIquestion{Calculer la fréquence de rafraîchissement maximale pour cette installation.}
\UPSTIquestion{Quel est le gain de temps (en \%) par rapport à une trame complète de 512 canaux ?}
\end{UPSTIexercice}

\begin{UPSTIprofOnlyEnv}
    \begin{UPSTIcorrectionP}{Optimisation du nombre de canaux}
        \textbf{1)} Durée avec 24 canaux (+ Start Code = 25 octets) :
        \begin{align*}
        T_{\text{octets}} &= 25 \times 44~\mu\text{s} = 1\,100~\mu\text{s} \\
        T_{\text{trame}} &= 176 + 12 + 1\,100 = \boxed{1\,288~\mu\text{s} = 1{,}288~\text{ms}}
        \end{align*}

        \textbf{2)} Fréquence maximale :
        \[
        f_{\text{max}} = \frac{1}{1{,}288 \times 10^{-3}} \approx \boxed{776{,}4~\text{Hz}}
        \]

        \textbf{3)} Gain de temps :
        \[
        \text{Gain} = \frac{22{,}76 - 1{,}288}{22{,}76} \times 100\% \approx \boxed{94{,}3\%}
        \]

        L'optimisation permet une fréquence de rafraîchissement environ \textbf{17 fois plus élevée} !
    \end{UPSTIcorrectionP}
\end{UPSTIprofOnlyEnv}
