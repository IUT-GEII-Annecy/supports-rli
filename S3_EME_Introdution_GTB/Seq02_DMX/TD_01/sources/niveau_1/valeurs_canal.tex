\begin{UPSTIexercice}{Signification des valeurs de canal}
Dans le contexte d'un projecteur avec variation d'intensité (dimmer), la valeur du canal DMX contrôle la luminosité.
\UPSTIquestion{Que signifie une valeur de canal égale à 0 ?}
\UPSTIquestion{Que signifie une valeur de canal égale à 255 ?}
\UPSTIquestion{Quelle valeur de canal correspond approximativement à 50\% d'intensité ?}
\UPSTIquestion{Pour un projecteur RGB (3 canaux : Rouge, Vert, Bleu), donner les valeurs des 3 canaux pour obtenir :}
\begin{itemize}
    \item Un rouge pur à pleine intensité
    \item Du blanc à pleine intensité
    \item Du jaune à pleine intensité
    \item De l'éclairage éteint
\end{itemize}
\end{UPSTIexercice}

\begin{UPSTIprofOnlyEnv}
    \begin{UPSTIcorrectionP}{Signification des valeurs de canal}
        \textbf{1)} Valeur 0 = \textbf{éteint} (0\% d'intensité)

        \textbf{2)} Valeur 255 = \textbf{pleine intensité} (100\%)

        \textbf{3)} 50\% d'intensité $\approx$ \textbf{127 ou 128} (255/2 = 127,5)

        \textbf{4)} Valeurs pour un projecteur RGB :
        \begin{center}
        \begin{tabular}{|l|c|c|c|}
        \hline
        \textbf{Couleur souhaitée} & \textbf{R} & \textbf{G} & \textbf{B} \\
        \hline
        Rouge pur & 255 & 0 & 0 \\
        Blanc & 255 & 255 & 255 \\
        Jaune (R+G) & 255 & 255 & 0 \\
        Éteint & 0 & 0 & 0 \\
        \hline
        \end{tabular}
        \end{center}
    \end{UPSTIcorrectionP}
\end{UPSTIprofOnlyEnv}
