\begin{UPSTIexercice}{Adressage d'appareils simples}
On installe 4 projecteurs LED RGB sur un bus DMX. Chaque projecteur utilise 3 canaux consécutifs (Rouge, Vert, Bleu).

Le premier projecteur (PAR 1) est configuré à l'adresse DMX 1.

\UPSTIquestion{Quels canaux DMX seront utilisés par le projecteur PAR 1 ?}
\UPSTIquestion{À quelle adresse doit-on configurer le projecteur PAR 2 pour qu'il utilise les canaux suivants ?}
\UPSTIquestion{Compléter le tableau d'adressage suivant :}

\begin{center}
\begin{tabular}{|c|c|c|c|}
\hline
\textbf{Projecteur} & \textbf{Adresse DMX} & \textbf{Canaux utilisés} & \textbf{Nombre de canaux} \\
\hline
PAR 1 & 1 & 1, 2, 3 & 3 \\
\hline
PAR 2 & ? & ? & 3 \\
\hline
PAR 3 & ? & ? & 3 \\
\hline
PAR 4 & ? & ? & 3 \\
\hline
\end{tabular}
\end{center}

\UPSTIquestion{Quel est le numéro du dernier canal utilisé dans cette installation ?}
\end{UPSTIexercice}

\begin{UPSTIprofOnlyEnv}
    \begin{UPSTIcorrectionP}{Adressage d'appareils simples}
        \textbf{1)} PAR 1 (adresse 1) utilise les canaux \textbf{1, 2 et 3} (R, G, B).

        \textbf{2)} PAR 2 doit être configuré à l'\textbf{adresse 4} (juste après les 3 canaux du PAR 1).

        \textbf{3)} Tableau complété :
        \begin{center}
        \begin{tabular}{|c|c|c|c|}
        \hline
        \textbf{Projecteur} & \textbf{Adresse DMX} & \textbf{Canaux utilisés} & \textbf{Nombre de canaux} \\
        \hline
        PAR 1 & 1 & 1, 2, 3 & 3 \\
        \hline
        PAR 2 & 4 & 4, 5, 6 & 3 \\
        \hline
        PAR 3 & 7 & 7, 8, 9 & 3 \\
        \hline
        PAR 4 & 10 & 10, 11, 12 & 3 \\
        \hline
        \end{tabular}
        \end{center}

        \textbf{4)} Le dernier canal utilisé est le \textbf{canal 12}.
    \end{UPSTIcorrectionP}
\end{UPSTIprofOnlyEnv}
