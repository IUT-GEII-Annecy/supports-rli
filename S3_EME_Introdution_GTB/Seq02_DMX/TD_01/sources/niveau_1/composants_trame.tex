\begin{UPSTIexercice}{Identifier les composants d'une trame}
On considère une trame DMX512 standard.
\UPSTIquestion{Lister dans l'ordre les différentes parties qui composent une trame DMX, de son début jusqu'au dernier canal transmis.}
\UPSTIquestion{Quelle est la valeur du Start Code pour une trame DMX standard contenant des données d'éclairage ?}
\UPSTIquestion{Combien de canaux peut-on transmettre au maximum dans une seule trame DMX ?}
\UPSTIquestion{Quelle est la plage de valeurs possibles pour un canal DMX ?}
\end{UPSTIexercice}

\begin{UPSTIprofOnlyEnv}
    \begin{UPSTIcorrectionP}{Identifier les composants d'une trame}
        \textbf{1)} Les parties d'une trame DMX dans l'ordre :
        \begin{itemize}
            \item BREAK (signal bas $\geq$ 88 $\mu$s)
            \item MAB - Mark After Break (signal haut $\geq$ 8 $\mu$s)
            \item Start Code (1 octet)
            \item Canal 1 (1 octet)
            \item Canal 2 (1 octet)
            \item ... (jusqu'à 512 canaux maximum)
        \end{itemize}

        \textbf{2)} Le Start Code standard est \textbf{0x00} (0 en décimal).

        \textbf{3)} On peut transmettre \textbf{512 canaux} maximum.

        \textbf{4)} Chaque canal est codé sur 1 octet : valeurs de \textbf{0 à 255}.
    \end{UPSTIcorrectionP}
\end{UPSTIprofOnlyEnv}
