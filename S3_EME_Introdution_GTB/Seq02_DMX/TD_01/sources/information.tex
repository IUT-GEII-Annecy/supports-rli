\section*{Introduction au protocole DMX512}

\subsection*{Contexte et objectifs}

Le protocole \textbf{DMX512} (Digital Multiplex) est un standard de communication utilisé principalement dans l'industrie du spectacle et de l'éclairage scénique pour contrôler des équipements comme :
\begin{itemize}
    \item Les projecteurs à variation d'intensité (dimmers)
    \item Les projecteurs asservis (lyres, scanners)
    \item Les machines à effets (fumée, stroboscopes, etc.)
    \item Les éclairages LED RGB/RGBA
\end{itemize}

\subsection*{Objectifs de ce TD}
À l'issue de ce TD, vous serez capable de :
\begin{itemize}
    \item Expliquer le principe de fonctionnement du protocole DMX512
    \item Décrire la structure d'une trame DMX
    \item Calculer les caractéristiques temporelles d'une transmission DMX
    \item Dimensionner un réseau DMX (câblage, connectique, terminaison)
    \item Diagnostiquer des problèmes simples sur une installation DMX
\end{itemize}

\pagebreak

\section*{Rappels théoriques}

\begin{UPSTIinfor}{Qu'est-ce que le DMX512 ?}
Le \textbf{DMX512} (USITT DMX512-A) est un protocole de communication série asynchrone unidirectionnel développé pour l'industrie du spectacle. Il permet de contrôler jusqu'à \textbf{512 canaux} (ou circuits) sur un seul bus.

\textbf{Caractéristiques principales :}
\begin{itemize}
    \item \textbf{Débit :} 250 kbits/s
    \item \textbf{Topologie :} Bus linéaire (chaîne)
    \item \textbf{Nombre de canaux :} 512 maximum par univers
    \item \textbf{Valeur d'un canal :} 0 à 255 (8 bits)
    \item \textbf{Support physique :} RS-485 (paire différentielle)
\end{itemize}
\end{UPSTIinfor}

\begin{UPSTIinfor}{La liaison série RS-485}
\label{info:rs485}
Le DMX512 utilise la couche physique \textbf{RS-485}, une norme de transmission série différentielle qui présente plusieurs avantages :

\begin{description}
    \item[Transmission différentielle :] Les données sont transmises sur deux fils (A et B ou Data+ et Data-). Le signal utile est la \textbf{différence de tension} entre ces deux fils, ce qui assure une bonne immunité au bruit.

    \item[Distance maximale :] Jusqu'à \textbf{400 mètres} pour le DMX (limité par la norme DMX512-A à 300-400m selon les sources).

    \item[Caractéristiques électriques :]
    \begin{itemize}
        \item Tension différentielle : $\pm$ 2V à $\pm$ 6V
        \item Impédance caractéristique : 120 $\Omega$
        \item Nécessite une résistance de terminaison de 120 $\Omega$ en bout de ligne
    \end{itemize}
\end{description}

\textbf{Note importante :} Bien que DMX utilise parfois des connecteurs XLR à 3 ou 5 broches (comme pour l'audio), il s'agit d'un signal numérique et non d'un signal audio. Ne jamais confondre les deux !
\end{UPSTIinfor}

\begin{UPSTIinfor}{Structure d'une trame DMX512}
\label{info:trame}
Une trame DMX est composée de plusieurs parties, transmises en série :

\begin{center}
\begin{tabular}{|c|c|c|c|c|c|c|}
\hline
\textbf{BREAK} & \textbf{MAB} & \textbf{Start Code} & \textbf{Canal 1} & \textbf{Canal 2} & \textbf{...} & \textbf{Canal 512} \\
\hline
$\geq 88~\mu$s & $\geq 8~\mu$s & 1 octet (0x00) & 1 octet & 1 octet & ... & 1 octet \\
\hline
\end{tabular}
\end{center}

\textbf{Description des éléments :}
\begin{description}
    \item[BREAK :] Signal bas (0 logique) d'au moins 88 $\mu$s (mais souvent 176 $\mu$s en pratique). Il marque le début d'une nouvelle trame.

    \item[MAB (Mark After Break) :] Signal haut (1 logique) d'au moins 8 $\mu$s (souvent 12 $\mu$s). C'est un temps de repos après le BREAK.

    \item[Start Code :] Un octet qui indique le type de données. Pour le DMX standard, c'est \textbf{0x00}. D'autres valeurs sont utilisées pour des extensions (RDM, texte, etc.).

    \item[Canaux 1 à 512 :] Chaque canal est un octet (0-255) représentant une valeur de contrôle. Tous les canaux n'ont pas besoin d'être transmis (voir exercices).
\end{description}

\textbf{Format d'un octet DMX :}
\begin{center}
\begin{tabular}{|c|c|c|c|c|c|c|c|c|c|c|}
\hline
\textbf{Start} & \textbf{D0} & \textbf{D1} & \textbf{D2} & \textbf{D3} & \textbf{D4} & \textbf{D5} & \textbf{D6} & \textbf{D7} & \textbf{Stop} & \textbf{Stop} \\
\hline
0 & \multicolumn{8}{c|}{8 bits de données (LSB first)} & 1 & 1 \\
\hline
\end{tabular}
\end{center}

Chaque octet est transmis avec :
\begin{itemize}
    \item 1 bit de START (0)
    \item 8 bits de données (bit de poids faible D0 en premier)
    \item 2 bits de STOP (1, 1)
\end{itemize}

\textbf{Durée de transmission d'un octet :}
À 250 kbits/s, avec 11 bits par octet : $\frac{11 \text{ bits}}{250\,000 \text{ bits/s}} = 44~\mu\text{s}$
\end{UPSTIinfor}

\begin{UPSTIinfor}{Câblage et connectique DMX}
\label{info:cablage}

\textbf{Câble :}
\begin{itemize}
    \item Paire torsadée blindée (STP - Shielded Twisted Pair)
    \item Impédance caractéristique : 120 $\Omega$ $\pm$ 20\%
    \item Section recommandée : 0,22 à 0,5 mm$^2$
    \item Blindage relié à la masse \textbf{uniquement côté émetteur}
\end{itemize}

\textbf{Connecteurs :}
\begin{itemize}
    \item \textbf{XLR 5 broches} (recommandé par la norme)
    \item XLR 3 broches (usage courant mais non recommandé)
\end{itemize}

\textbf{Brochage XLR 5 broches :}
\begin{center}
\begin{tabular}{|c|l|}
\hline
\textbf{Broche} & \textbf{Fonction} \\
\hline
1 & Masse / Blindage (GND) \\
2 & Data- (DMX-) \\
3 & Data+ (DMX+) \\
4 & Réservé (seconde paire) \\
5 & Réservé (seconde paire) \\
\hline
\end{tabular}
\end{center}

\textbf{Topologie :}
\begin{itemize}
    \item Architecture en \textbf{chaîne} (daisy chain) : émetteur $\to$ appareil 1 $\to$ appareil 2 $\to$ ... $\to$ appareil N
    \item \textbf{Pas de dérivation en T} (star ou T-tap) : chaque appareil doit avoir une entrée DMX IN et une sortie DMX OUT
    \item Nombre maximal d'appareils : \textbf{32 récepteurs} par univers (limitation électrique du RS-485)
    \item Pour plus de 32 appareils : utiliser un \textbf{splitter DMX} (répéteur/amplificateur)
\end{itemize}

\textbf{Terminaison :}
\begin{itemize}
    \item Une résistance de \textbf{120 $\Omega$} doit être placée \textbf{en bout de ligne} (entre Data+ et Data-)
    \item Cette résistance évite les réflexions de signal qui peuvent causer des erreurs de transmission
    \item Certains appareils ont une terminaison intégrée (switch DIP ou menu)
\end{itemize}
\end{UPSTIinfor}

\begin{UPSTIinfor}{Adressage DMX}
\label{info:adressage}

Chaque appareil DMX possède une \textbf{adresse de départ} (start address) qui détermine quels canaux du bus DMX il va écouter.

\textbf{Exemples :}
\begin{itemize}
    \item Un \textbf{dimmer simple} (1 canal pour l'intensité) à l'adresse 1 écoute le canal 1
    \item Un \textbf{projecteur RGB} (3 canaux : R, G, B) à l'adresse 10 écoute les canaux 10, 11 et 12
    \item Une \textbf{lyre} (16 canaux : pan, tilt, couleur, gobo, etc.) à l'adresse 100 écoute les canaux 100 à 115
\end{itemize}

\textbf{Calcul du nombre d'appareils :}
Si on a des appareils utilisant respectivement $n_1, n_2, \ldots, n_k$ canaux, il faut vérifier que :
\[
n_1 + n_2 + \cdots + n_k \leq 512
\]

\textbf{Plan d'adressage :}
Il est essentiel de documenter l'adressage de chaque appareil pour faciliter la programmation et la maintenance. Un tableau typique contient :
\begin{center}
\begin{tabular}{|c|c|c|c|}
\hline
\textbf{Appareil} & \textbf{Type} & \textbf{Adresse DMX} & \textbf{Nb canaux} \\
\hline
PAR LED 1 & RGB & 1 & 3 \\
PAR LED 2 & RGB & 4 & 3 \\
Lyre 1 & 16 canaux & 10 & 16 \\
\hline
\end{tabular}
\end{center}
\end{UPSTIinfor}

\begin{UPSTIinfor}{Fréquence de rafraîchissement}
\label{info:refresh}

Le contrôleur DMX envoie des trames en continu, même si les valeurs ne changent pas. C'est le principe du \textbf{rafraîchissement}.

\textbf{Fréquence typique :}
\begin{itemize}
    \item Minimum selon la norme : 1 Hz (1 trame par seconde)
    \item Typique en pratique : 25 à 44 Hz (40 à 25 ms entre deux trames)
\end{itemize}

\textbf{Calcul de la durée d'une trame complète :}
\begin{align*}
T_{\text{trame}} &= T_{\text{BREAK}} + T_{\text{MAB}} + T_{\text{Start Code}} + T_{\text{canaux}} \\
&= 176~\mu\text{s} + 12~\mu\text{s} + 44~\mu\text{s} + (N \times 44~\mu\text{s})
\end{align*}

où $N$ est le nombre de canaux transmis (de 1 à 512).

\textbf{Exemple :} Pour 512 canaux :
\begin{align*}
T_{\text{trame}} &= 176 + 12 + 44 + (512 \times 44) \\
&= 232 + 22\,528 = 22\,760~\mu\text{s} \approx 22{,}76~\text{ms}
\end{align*}

La fréquence maximale théorique est donc : $f_{\text{max}} = \frac{1}{22{,}76~\text{ms}} \approx 44~\text{Hz}$
\end{UPSTIinfor}
