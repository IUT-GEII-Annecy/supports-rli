\begin{UPSTIexercice}{DMX sur Ethernet (optionnel)}
Pour les grandes installations, le DMX peut être transporté sur des réseaux Ethernet.

\UPSTIquestion{Citer deux protocoles permettant de transporter du DMX sur Ethernet.}
\UPSTIquestion{Quels sont les avantages de ces solutions par rapport au DMX filaire traditionnel ?}
\UPSTIquestion{Rechercher ce qu'est un "nœud DMX" (DMX node) dans ce contexte.}
\end{UPSTIexercice}

\begin{UPSTIprofOnlyEnv}
    \begin{UPSTIcorrectionP}{DMX sur Ethernet (optionnel)}
        \textbf{1)} Deux protocoles courants :
        \begin{itemize}
            \item \textbf{Art-Net} (protocole ouvert, très répandu)
            \item \textbf{sACN} (streaming ACN, ANSI E1.31, standard ESTA)
        \end{itemize}

        \textbf{2)} Avantages :
        \begin{itemize}
            \item Distribution sur de très longues distances (limitation : réseau Ethernet)
            \item Possibilité de transporter plusieurs univers DMX sur un seul câble
            \item Infrastructure réseau standard (switches, câbles Cat5e/Cat6)
            \item Possibilité de contrôle via réseau IP (WiFi, Internet)
            \item Monitoring et diagnostic via le réseau
        \end{itemize}

        \textbf{3)} Un \textbf{nœud DMX} (DMX node ou gateway) est un appareil qui convertit les données DMX transportées sur Ethernet (Art-Net, sACN) en signaux DMX512 classiques (RS-485). Il possède généralement plusieurs sorties DMX (univers) et une connexion Ethernet.
    \end{UPSTIcorrectionP}
\end{UPSTIprofOnlyEnv}
