\begin{UPSTIexercice}{RDM (Remote Device Management)}
Le protocole RDM est une extension du DMX512 qui permet une communication \textbf{bidirectionnelle}.

\UPSTIquestion{Rechercher et expliquer brièvement ce qu'est le protocole RDM.}
\UPSTIquestion{Quels sont les avantages du RDM par rapport au DMX standard ?}
\UPSTIquestion{Quel est le Start Code utilisé pour les messages RDM ?}
\UPSTIquestion{Le RDM est-il rétrocompatible avec le DMX standard ? Expliquer.}
\end{UPSTIexercice}

\begin{UPSTIprofOnlyEnv}
    \begin{UPSTIcorrectionP}{RDM (Remote Device Management)}
        \textbf{1)} Le RDM (ANSI E1.20) est une extension du DMX512 qui ajoute une communication bidirectionnelle. Il permet au contrôleur d'interroger les appareils et de recevoir des informations en retour.

        \textbf{2)} Avantages du RDM :
        \begin{itemize}
            \item Configuration à distance (adressage automatique)
            \item Diagnostic et monitoring (température, heures d'utilisation, erreurs)
            \item Mise à jour de firmware à distance
            \item Récupération d'informations (modèle, numéro de série, modes disponibles)
        \end{itemize}

        \textbf{3)} Le RDM utilise le Start Code \textbf{0xCC} (204 en décimal) pour ses messages.

        \textbf{4)} Oui, le RDM est \textbf{rétrocompatible}. Les trames DMX standard (Start Code 0x00) continuent de fonctionner normalement. Les appareils non-RDM ignorent simplement les trames avec Start Code 0xCC. Les deux types de trames peuvent coexister sur le même bus.
    \end{UPSTIcorrectionP}
\end{UPSTIprofOnlyEnv}
