\begin{UPSTIexercice}{Étude de cas : Festival en plein air}
Un organisateur de festival souhaite installer un système d'éclairage pour une scène extérieure. L'installation comprend :

\textbf{Équipements :}
\begin{itemize}
    \item 32 projecteurs PAR LED RGB (3 canaux chacun) répartis sur la structure de scène
    \item 6 lyres (18 canaux chacune) en fond de scène
    \item 8 stroboscopes (1 canal chacun) sur les côtés
    \item 2 lasers (8 canaux chacun)
\end{itemize}

\textbf{Contraintes :}
\begin{itemize}
    \item Le contrôleur DMX est installé en régie, à 85 mètres de la scène
    \item Les projecteurs PAR sont répartis en 4 groupes de 8 projecteurs chacun
    \item Le budget impose l'utilisation d'un seul contrôleur DMX (2 univers disponibles)
\end{itemize}

\UPSTIquestion{Calculer le nombre total de canaux DMX nécessaires.}

\UPSTIquestion{Proposer une répartition des appareils sur 2 univers en justifiant vos choix (facilité de programmation, câblage, etc.).}

\UPSTIquestion{Pour l'univers 1, établir un plan d'adressage détaillé en regroupant les PAR LED par groupe.}

\UPSTIquestion{Calculer la durée d'une trame DMX pour chaque univers (en considérant que seuls les canaux utilisés sont transmis).}

\UPSTIquestion{En déduire la fréquence de rafraîchissement maximale pour chaque univers.}

\UPSTIquestion{La distance de 85 mètres entre la régie et la scène est-elle compatible avec le DMX ? Quelles précautions doit-on prendre ?}

\UPSTIquestion{Proposer un schéma de câblage pour minimiser la longueur de câble nécessaire, sachant que les 4 groupes de PAR sont aux 4 coins de la scène, les lyres en fond de scène et les stroboscopes sur les côtés.}
\end{UPSTIexercice}

\begin{UPSTIprofOnlyEnv}
    \begin{UPSTIcorrectionP}{Étude de cas : Festival en plein air}
        \textbf{1)} Nombre total de canaux :
        \begin{align*}
        N_{\text{total}} &= (32 \times 3) + (6 \times 18) + (8 \times 1) + (2 \times 8) \\
        &= 96 + 108 + 8 + 16 \\
        &= \boxed{228~\text{canaux}}
        \end{align*}

        \textbf{2)} Répartition sur 2 univers (proposition) :

        \textbf{Univers 1 :} Effets statiques et dynamiques (132 canaux)
        \begin{itemize}
            \item 32 PAR LED RGB : $32 \times 3 = 96$ canaux
            \item 6 Lyres : $6 \times 18 = 108$ canaux $\to$ \textbf{Trop ! Mettre 3 lyres seulement = 54 canaux}
            \item Total Univers 1 : $96 + 54 = 150$ canaux
        \end{itemize}

        \textbf{Univers 2 :} Compléments et effets spéciaux (78 canaux)
        \begin{itemize}
            \item 3 Lyres restantes : $3 \times 18 = 54$ canaux
            \item 8 Stroboscopes : $8 \times 1 = 8$ canaux
            \item 2 Lasers : $2 \times 8 = 16$ canaux
            \item Total Univers 2 : $54 + 8 + 16 = 78$ canaux
        \end{itemize}

        \textit{Justification :} Cette répartition sépare les PAR (wash général) des effets mobiles, facilite la programmation et équilibre la charge entre les deux univers.

        \textbf{3)} Plan d'adressage Univers 1 :

        \begin{center}
        \scriptsize
        \begin{tabular}{|c|c|c|}
        \hline
        \textbf{Appareil} & \textbf{Adresse} & \textbf{Canaux} \\
        \hline
        \multicolumn{3}{|c|}{\textbf{Groupe 1 - Jardin avant}} \\
        \hline
        PAR 1 à 8 & 1, 4, 7, ..., 22 & 1-24 \\
        \hline
        \multicolumn{3}{|c|}{\textbf{Groupe 2 - Cour avant}} \\
        \hline
        PAR 9 à 16 & 25, 28, 31, ..., 46 & 25-48 \\
        \hline
        \multicolumn{3}{|c|}{\textbf{Groupe 3 - Jardin arrière}} \\
        \hline
        PAR 17 à 24 & 49, 52, 55, ..., 70 & 49-72 \\
        \hline
        \multicolumn{3}{|c|}{\textbf{Groupe 4 - Cour arrière}} \\
        \hline
        PAR 25 à 32 & 73, 76, 79, ..., 94 & 73-96 \\
        \hline
        \multicolumn{3}{|c|}{\textbf{Lyres (3 premières)}} \\
        \hline
        Lyre 1 & 97 & 97-114 \\
        Lyre 2 & 115 & 115-132 \\
        Lyre 3 & 133 & 133-150 \\
        \hline
        \end{tabular}
        \end{center}

        \textbf{4)} Durée des trames :

        \textit{Univers 1 (150 canaux + Start Code = 151 octets) :}
        \begin{align*}
        T_1 &= 176 + 12 + (151 \times 44) = 188 + 6\,644 = \boxed{6\,832~\mu\text{s} = 6{,}83~\text{ms}}
        \end{align*}

        \textit{Univers 2 (78 canaux + Start Code = 79 octets) :}
        \begin{align*}
        T_2 &= 176 + 12 + (79 \times 44) = 188 + 3\,476 = \boxed{3\,664~\mu\text{s} = 3{,}66~\text{ms}}
        \end{align*}

        \textbf{5)} Fréquences de rafraîchissement :
        \[
        f_1 = \frac{1}{6{,}83 \times 10^{-3}} \approx \boxed{146~\text{Hz}}
        \]
        \[
        f_2 = \frac{1}{3{,}66 \times 10^{-3}} \approx \boxed{273~\text{Hz}}
        \]

        \textbf{6)} Distance de 85 m : \textbf{Compatible} (limite : 300-400 m).

        \textit{Précautions :}
        \begin{itemize}
            \item Utiliser un câble de qualité (paire torsadée blindée, 120 $\Omega$)
            \item Installer une résistance de terminaison en bout de ligne
            \item Protéger le câble des intempéries et du passage du public
            \item Prévoir un chemin de câble dédié, séparé des alimentations électriques 230V
        \end{itemize}

        \textbf{7)} Schéma de câblage proposé :

        \textit{Stratégie :} Créer une chaîne qui minimise les allers-retours.

        Exemple de parcours : Régie $\to$ Groupe 1 (Jardin avant) $\to$ Groupe 3 (Jardin arrière) $\to$ Lyres (fond de scène) $\to$ Groupe 4 (Cour arrière) $\to$ Groupe 2 (Cour avant) $\to$ retour.

        Alternative : Utiliser un splitter DMX au niveau de la scène pour créer plusieurs branches et réduire la longueur totale de câble.
    \end{UPSTIcorrectionP}
\end{UPSTIprofOnlyEnv}
