\setcounter{tocdepth}{2}
\tableofcontents
\pagebreak
\section*{Introduction au protocole DMX512}

\subsection*{Contexte et objectifs}

Le protocole \textbf{DMX512} (Digital Multiplex) est un standard de communication utilisé principalement dans l'industrie du spectacle et de l'éclairage scénique pour contrôler des équipements comme :
\begin{itemize}
    \item Les projecteurs à variation d'intensité (dimmers)
    \item Les projecteurs asservis (lyres, scanners)
    \item Les machines à effets (fumée, stroboscopes, etc.)
    \item Les éclairages LED RGB/RGBA
\end{itemize}

\subsection*{Objectifs de ce TD}
À l'issue de ce TD, vous serez capable de :
\begin{itemize}
    \item Expliquer le principe de fonctionnement du protocole DMX512
    \item Décrire la structure d'une trame DMX
    \item Calculer les caractéristiques temporelles d'une transmission DMX
    \item Dimensionner un réseau DMX (câblage, connectique, terminaison)
    \item Diagnostiquer des problèmes simples sur une installation DMX
\end{itemize}

\pagebreak

\section*{Rappels théoriques}

\begin{UPSTIinfor}{Qu'est-ce que le DMX512 ?}
Le \textbf{DMX512} (USITT DMX512-A) est un protocole de communication série asynchrone unidirectionnel développé pour l'industrie du spectacle. Il permet de contrôler jusqu'à \textbf{512 canaux} (ou circuits) sur un seul bus.

\textbf{Caractéristiques principales :}
\begin{itemize}
    \item \textbf{Débit :} 250 kbits/s
    \item \textbf{Topologie :} Bus linéaire (chaîne)
    \item \textbf{Nombre de canaux :} 512 maximum par univers
    \item \textbf{Valeur d'un canal :} 0 à 255 (8 bits)
    \item \textbf{Support physique :} RS-485 (paire différentielle)
\end{itemize}
\end{UPSTIinfor}

\begin{UPSTIinfor}{La liaison série RS-485}
\label{info:rs485}
Le DMX512 utilise la couche physique \textbf{RS-485}, une norme de transmission série différentielle qui présente plusieurs avantages :

\begin{description}
    \item[Transmission différentielle :] Les données sont transmises sur deux fils (A et B ou Data+ et Data-). Le signal utile est la \textbf{différence de tension} entre ces deux fils, ce qui assure une bonne immunité au bruit.

    \item[Distance maximale :] Jusqu'à \textbf{400 mètres} pour le DMX (limité par la norme DMX512-A à 300-400m selon les sources).

    \item[Caractéristiques électriques :]
    \begin{itemize}
        \item Tension différentielle : $\pm$ 2V à $\pm$ 6V
        \item Impédance caractéristique : 120 $\Omega$
        \item Nécessite une résistance de terminaison de 120 $\Omega$ en bout de ligne
    \end{itemize}
\end{description}

\textbf{Note importante :} Bien que DMX utilise parfois des connecteurs XLR à 3 ou 5 broches (comme pour l'audio), il s'agit d'un signal numérique et non d'un signal audio. Ne jamais confondre les deux !
\end{UPSTIinfor}

\begin{UPSTIinfor}{Structure d'une trame DMX512}
\label{info:trame}
Une trame DMX est composée de plusieurs parties, transmises en série :

\begin{center}
\begin{tabular}{|c|c|c|c|c|c|c|}
\hline
\textbf{BREAK} & \textbf{MAB} & \textbf{Start Code} & \textbf{Canal 1} & \textbf{Canal 2} & \textbf{...} & \textbf{Canal 512} \\
\hline
$\geq 88~\mu$s & $\geq 8~\mu$s & 1 octet (0x00) & 1 octet & 1 octet & ... & 1 octet \\
\hline
\end{tabular}
\end{center}

\textbf{Description des éléments :}
\begin{description}
    \item[BREAK :] Signal bas (0 logique) d'au moins 88 $\mu$s (mais souvent 176 $\mu$s en pratique). Il marque le début d'une nouvelle trame.

    \item[MAB (Mark After Break) :] Signal haut (1 logique) d'au moins 8 $\mu$s (souvent 12 $\mu$s). C'est un temps de repos après le BREAK.

    \item[Start Code :] Un octet qui indique le type de données. Pour le DMX standard, c'est \textbf{0x00}. D'autres valeurs sont utilisées pour des extensions (RDM, texte, etc.).

    \item[Canaux 1 à 512 :] Chaque canal est un octet (0-255) représentant une valeur de contrôle. Tous les canaux n'ont pas besoin d'être transmis (voir exercices).
\end{description}

\textbf{Format d'un octet DMX :}
\begin{center}
\begin{tabular}{|c|c|c|c|c|c|c|c|c|c|c|}
\hline
\textbf{Start} & \textbf{D0} & \textbf{D1} & \textbf{D2} & \textbf{D3} & \textbf{D4} & \textbf{D5} & \textbf{D6} & \textbf{D7} & \textbf{Stop} & \textbf{Stop} \\
\hline
0 & \multicolumn{8}{c|}{8 bits de données (LSB first)} & 1 & 1 \\
\hline
\end{tabular}
\end{center}

Chaque octet est transmis avec :
\begin{itemize}
    \item 1 bit de START (0)
    \item 8 bits de données (bit de poids faible D0 en premier)
    \item 2 bits de STOP (1, 1)
\end{itemize}

\textbf{Durée de transmission d'un octet :}
À 250 kbits/s, avec 11 bits par octet : $\frac{11 \text{ bits}}{250\,000 \text{ bits/s}} = 44~\mu\text{s}$
\end{UPSTIinfor}

\begin{UPSTIinfor}{Câblage et connectique DMX}
\label{info:cablage}

\textbf{Câble :}
\begin{itemize}
    \item Paire torsadée blindée (STP - Shielded Twisted Pair)
    \item Impédance caractéristique : 120 $\Omega$ $\pm$ 20\%
    \item Section recommandée : 0,22 à 0,5 mm$^2$
    \item Blindage relié à la masse \textbf{uniquement côté émetteur}
\end{itemize}

\textbf{Connecteurs :}
\begin{itemize}
    \item \textbf{XLR 5 broches} (recommandé par la norme)
    \item XLR 3 broches (usage courant mais non recommandé)
\end{itemize}

\textbf{Brochage XLR 5 broches :}
\begin{center}
\begin{tabular}{|c|l|}
\hline
\textbf{Broche} & \textbf{Fonction} \\
\hline
1 & Masse / Blindage (GND) \\
2 & Data- (DMX-) \\
3 & Data+ (DMX+) \\
4 & Réservé (seconde paire) \\
5 & Réservé (seconde paire) \\
\hline
\end{tabular}
\end{center}

\textbf{Topologie :}
\begin{itemize}
    \item Architecture en \textbf{chaîne} (daisy chain) : émetteur $\to$ appareil 1 $\to$ appareil 2 $\to$ ... $\to$ appareil N
    \item \textbf{Pas de dérivation en T} (star ou T-tap) : chaque appareil doit avoir une entrée DMX IN et une sortie DMX OUT
    \item Nombre maximal d'appareils : \textbf{32 récepteurs} par univers (limitation électrique du RS-485)
    \item Pour plus de 32 appareils : utiliser un \textbf{splitter DMX} (répéteur/amplificateur)
\end{itemize}

\textbf{Terminaison :}
\begin{itemize}
    \item Une résistance de \textbf{120 $\Omega$} doit être placée \textbf{en bout de ligne} (entre Data+ et Data-)
    \item Cette résistance évite les réflexions de signal qui peuvent causer des erreurs de transmission
    \item Certains appareils ont une terminaison intégrée (switch DIP ou menu)
\end{itemize}
\end{UPSTIinfor}

\begin{UPSTIinfor}{Adressage DMX}
\label{info:adressage}

Chaque appareil DMX possède une \textbf{adresse de départ} (start address) qui détermine quels canaux du bus DMX il va écouter.

\textbf{Exemples :}
\begin{itemize}
    \item Un \textbf{dimmer simple} (1 canal pour l'intensité) à l'adresse 1 écoute le canal 1
    \item Un \textbf{projecteur RGB} (3 canaux : R, G, B) à l'adresse 10 écoute les canaux 10, 11 et 12
    \item Une \textbf{lyre} (16 canaux : pan, tilt, couleur, gobo, etc.) à l'adresse 100 écoute les canaux 100 à 115
\end{itemize}

\textbf{Calcul du nombre d'appareils :}
Si on a des appareils utilisant respectivement $n_1, n_2, \ldots, n_k$ canaux, il faut vérifier que :
\[
n_1 + n_2 + \cdots + n_k \leq 512
\]

\textbf{Plan d'adressage :}
Il est essentiel de documenter l'adressage de chaque appareil pour faciliter la programmation et la maintenance. Un tableau typique contient :
\begin{center}
\begin{tabular}{|c|c|c|c|}
\hline
\textbf{Appareil} & \textbf{Type} & \textbf{Adresse DMX} & \textbf{Nb canaux} \\
\hline
PAR LED 1 & RGB & 1 & 3 \\
PAR LED 2 & RGB & 4 & 3 \\
Lyre 1 & 16 canaux & 10 & 16 \\
\hline
\end{tabular}
\end{center}
\end{UPSTIinfor}

\begin{UPSTIinfor}{Fréquence de rafraîchissement}
\label{info:refresh}

Le contrôleur DMX envoie des trames en continu, même si les valeurs ne changent pas. C'est le principe du \textbf{rafraîchissement}.

\textbf{Fréquence typique :}
\begin{itemize}
    \item Minimum selon la norme : 1 Hz (1 trame par seconde)
    \item Typique en pratique : 25 à 44 Hz (40 à 25 ms entre deux trames)
\end{itemize}

\textbf{Calcul de la durée d'une trame complète :}
\begin{align*}
T_{\text{trame}} &= T_{\text{BREAK}} + T_{\text{MAB}} + T_{\text{Start Code}} + T_{\text{canaux}} \\
&= 176~\mu\text{s} + 12~\mu\text{s} + 44~\mu\text{s} + (N \times 44~\mu\text{s})
\end{align*}

où $N$ est le nombre de canaux transmis (de 1 à 512).

\textbf{Exemple :} Pour 512 canaux :
\begin{align*}
T_{\text{trame}} &= 176 + 12 + 44 + (512 \times 44) \\
&= 232 + 22\,528 = 22\,760~\mu\text{s} \approx 22{,}76~\text{ms}
\end{align*}

La fréquence maximale théorique est donc : $f_{\text{max}} = \frac{1}{22{,}76~\text{ms}} \approx 44~\text{Hz}$
\end{UPSTIinfor}

\pagebreak

\section{Niveau 1 - Découverte du protocole DMX512}

\subsection{Structure d'une trame DMX}
\begin{UPSTIexercice}{Identifier les composants d'une trame}
On considère une trame DMX512 standard.
\UPSTIquestion{Lister dans l'ordre les différentes parties qui composent une trame DMX, de son début jusqu'au dernier canal transmis.}
\UPSTIquestion{Quelle est la valeur du Start Code pour une trame DMX standard contenant des données d'éclairage ?}
\UPSTIquestion{Combien de canaux peut-on transmettre au maximum dans une seule trame DMX ?}
\UPSTIquestion{Quelle est la plage de valeurs possibles pour un canal DMX ?}
\end{UPSTIexercice}

\begin{UPSTIprofOnlyEnv}
    \begin{UPSTIcorrectionP}{Identifier les composants d'une trame}
        \textbf{1)} Les parties d'une trame DMX dans l'ordre :
        \begin{itemize}
            \item BREAK (signal bas $\geq$ 88 $\mu$s)
            \item MAB - Mark After Break (signal haut $\geq$ 8 $\mu$s)
            \item Start Code (1 octet)
            \item Canal 1 (1 octet)
            \item Canal 2 (1 octet)
            \item ... (jusqu'à 512 canaux maximum)
        \end{itemize}

        \textbf{2)} Le Start Code standard est \textbf{0x00} (0 en décimal).

        \textbf{3)} On peut transmettre \textbf{512 canaux} maximum.

        \textbf{4)} Chaque canal est codé sur 1 octet : valeurs de \textbf{0 à 255}.
    \end{UPSTIcorrectionP}
\end{UPSTIprofOnlyEnv}

\begin{UPSTIexercice}{Signification des valeurs de canal}
Dans le contexte d'un projecteur avec variation d'intensité (dimmer), la valeur du canal DMX contrôle la luminosité.
\UPSTIquestion{Que signifie une valeur de canal égale à 0 ?}
\UPSTIquestion{Que signifie une valeur de canal égale à 255 ?}
\UPSTIquestion{Quelle valeur de canal correspond approximativement à 50\% d'intensité ?}
\UPSTIquestion{Pour un projecteur RGB (3 canaux : Rouge, Vert, Bleu), donner les valeurs des 3 canaux pour obtenir :}
\begin{itemize}
    \item Un rouge pur à pleine intensité
    \item Du blanc à pleine intensité
    \item Du jaune à pleine intensité
    \item De l'éclairage éteint
\end{itemize}
\end{UPSTIexercice}

\begin{UPSTIprofOnlyEnv}
    \begin{UPSTIcorrectionP}{Signification des valeurs de canal}
        \textbf{1)} Valeur 0 = \textbf{éteint} (0\% d'intensité)

        \textbf{2)} Valeur 255 = \textbf{pleine intensité} (100\%)

        \textbf{3)} 50\% d'intensité $\approx$ \textbf{127 ou 128} (255/2 = 127,5)

        \textbf{4)} Valeurs pour un projecteur RGB :
        \begin{center}
        \begin{tabular}{|l|c|c|c|}
        \hline
        \textbf{Couleur souhaitée} & \textbf{R} & \textbf{G} & \textbf{B} \\
        \hline
        Rouge pur & 255 & 0 & 0 \\
        Blanc & 255 & 255 & 255 \\
        Jaune (R+G) & 255 & 255 & 0 \\
        Éteint & 0 & 0 & 0 \\
        \hline
        \end{tabular}
        \end{center}
    \end{UPSTIcorrectionP}
\end{UPSTIprofOnlyEnv}


\subsection{Calculs temporels de base}
\begin{UPSTIexercice}{Durée de transmission d'un octet}
Le protocole DMX512 utilise une vitesse de transmission de 250 kbits/s. Chaque octet DMX est transmis avec 11 bits (1 start + 8 données + 2 stop).

\UPSTIquestion{Rappeler la relation entre le débit binaire $D$ (en bits/s), le temps de transmission $T$ (en secondes) et le nombre de bits $n$.}
\UPSTIquestion{Calculer la durée nécessaire pour transmettre 1 bit à 250 kbits/s.}
\UPSTIquestion{En déduire la durée de transmission d'un octet DMX (11 bits).}
\UPSTIquestion{Vérifier que ce résultat est cohérent avec la valeur donnée dans l'encart théorique page~\pageref{info:trame}.}
\end{UPSTIexercice}

\begin{UPSTIprofOnlyEnv}
    \begin{UPSTIcorrectionP}{Durée de transmission d'un octet}
        \textbf{1)} Relation : $T = \dfrac{n}{D}$ où $n$ est le nombre de bits et $D$ le débit en bits/s.

        \textbf{2)} Durée pour 1 bit :
        \[
        T_{\text{bit}} = \frac{1}{250\,000~\text{bits/s}} = 4 \times 10^{-6}~\text{s} = 4~\mu\text{s}
        \]

        \textbf{3)} Durée pour 1 octet DMX (11 bits) :
        \[
        T_{\text{octet}} = 11 \times T_{\text{bit}} = 11 \times 4~\mu\text{s} = 44~\mu\text{s}
        \]

        \textbf{4)} Ce résultat est \textbf{cohérent} avec la valeur de 44 $\mu$s indiquée page~\pageref{info:trame}.
    \end{UPSTIcorrectionP}
\end{UPSTIprofOnlyEnv}

\begin{UPSTIexercice}{Durée d'une trame DMX simplifiée}
On souhaite calculer la durée d'une trame DMX transmettant \textbf{8 canaux} (en plus du Start Code).

On considère les valeurs typiques suivantes :
\begin{itemize}
    \item BREAK : 176 $\mu$s
    \item MAB : 12 $\mu$s
    \item Durée d'un octet : 44 $\mu$s
\end{itemize}

\UPSTIquestion{Combien d'octets sont transmis au total dans cette trame ? (Penser au Start Code)}
\UPSTIquestion{Calculer la durée totale de transmission de tous les octets (Start Code + 8 canaux).}
\UPSTIquestion{Calculer la durée totale de la trame complète (BREAK + MAB + octets).}
\UPSTIquestion{Exprimer ce résultat en millisecondes.}
\end{UPSTIexercice}

\begin{UPSTIprofOnlyEnv}
    \begin{UPSTIcorrectionP}{Durée d'une trame DMX simplifiée}
        \textbf{1)} Nombre d'octets : 1 Start Code + 8 canaux = \textbf{9 octets}

        \textbf{2)} Durée de transmission des octets :
        \[
        T_{\text{octets}} = 9 \times 44~\mu\text{s} = 396~\mu\text{s}
        \]

        \textbf{3)} Durée totale de la trame :
        \begin{align*}
        T_{\text{trame}} &= T_{\text{BREAK}} + T_{\text{MAB}} + T_{\text{octets}} \\
        &= 176~\mu\text{s} + 12~\mu\text{s} + 396~\mu\text{s} \\
        &= 584~\mu\text{s}
        \end{align*}

        \textbf{4)} En millisecondes : $T_{\text{trame}} = 584~\mu\text{s} = \boxed{0{,}584~\text{ms}}$
    \end{UPSTIcorrectionP}
\end{UPSTIprofOnlyEnv}


\subsection{Adressage DMX}
\begin{UPSTIexercice}{Adressage d'appareils simples}
On installe 4 projecteurs LED RGB sur un bus DMX. Chaque projecteur utilise 3 canaux consécutifs (Rouge, Vert, Bleu).

Le premier projecteur (PAR 1) est configuré à l'adresse DMX 1.

\UPSTIquestion{Quels canaux DMX seront utilisés par le projecteur PAR 1 ?}
\UPSTIquestion{À quelle adresse doit-on configurer le projecteur PAR 2 pour qu'il utilise les canaux suivants ?}
\UPSTIquestion{Compléter le tableau d'adressage suivant :}

\begin{center}
\begin{tabular}{|c|c|c|c|}
\hline
\textbf{Projecteur} & \textbf{Adresse DMX} & \textbf{Canaux utilisés} & \textbf{Nombre de canaux} \\
\hline
PAR 1 & 1 & 1, 2, 3 & 3 \\
\hline
PAR 2 & ? & ? & 3 \\
\hline
PAR 3 & ? & ? & 3 \\
\hline
PAR 4 & ? & ? & 3 \\
\hline
\end{tabular}
\end{center}

\UPSTIquestion{Quel est le numéro du dernier canal utilisé dans cette installation ?}
\end{UPSTIexercice}

\begin{UPSTIprofOnlyEnv}
    \begin{UPSTIcorrectionP}{Adressage d'appareils simples}
        \textbf{1)} PAR 1 (adresse 1) utilise les canaux \textbf{1, 2 et 3} (R, G, B).

        \textbf{2)} PAR 2 doit être configuré à l'\textbf{adresse 4} (juste après les 3 canaux du PAR 1).

        \textbf{3)} Tableau complété :
        \begin{center}
        \begin{tabular}{|c|c|c|c|}
        \hline
        \textbf{Projecteur} & \textbf{Adresse DMX} & \textbf{Canaux utilisés} & \textbf{Nombre de canaux} \\
        \hline
        PAR 1 & 1 & 1, 2, 3 & 3 \\
        \hline
        PAR 2 & 4 & 4, 5, 6 & 3 \\
        \hline
        PAR 3 & 7 & 7, 8, 9 & 3 \\
        \hline
        PAR 4 & 10 & 10, 11, 12 & 3 \\
        \hline
        \end{tabular}
        \end{center}

        \textbf{4)} Le dernier canal utilisé est le \textbf{canal 12}.
    \end{UPSTIcorrectionP}
\end{UPSTIprofOnlyEnv}

\begin{UPSTIexercice}{Nombre maximal d'appareils}
On dispose d'un univers DMX (512 canaux) que l'on souhaite utiliser pour installer des projecteurs LED RGBW (4 canaux chacun : Rouge, Vert, Bleu, Blanc).

\UPSTIquestion{Combien de projecteurs RGBW peut-on installer au maximum sur cet univers DMX ?}
\UPSTIquestion{Justifier votre réponse par un calcul.}
\UPSTIquestion{Quel sera le numéro du dernier canal utilisé ?}
\UPSTIquestion{Combien de canaux resteront inutilisés ?}
\end{UPSTIexercice}

\begin{UPSTIprofOnlyEnv}
    \begin{UPSTIcorrectionP}{Nombre maximal d'appareils}
        \textbf{1)} Calcul :
        \[
        N_{\text{max}} = \left\lfloor \frac{512~\text{canaux}}{4~\text{canaux/projecteur}} \right\rfloor = \left\lfloor 128 \right\rfloor = \boxed{128~\text{projecteurs}}
        \]

        \textbf{2)} On divise le nombre total de canaux (512) par le nombre de canaux par appareil (4). La partie entière donne le nombre maximal d'appareils.

        \textbf{3)} Dernier canal utilisé :
        \[
        \text{Dernier canal} = 128 \times 4 = \boxed{512}
        \]

        \textbf{4)} Canaux inutilisés : $512 - 512 = \boxed{0~\text{canal}}$ (tous les canaux sont utilisés)
    \end{UPSTIcorrectionP}
\end{UPSTIprofOnlyEnv}


\pagebreak

\section{Niveau 2 - Approfondissement}

\subsection{Calculs temporels avancés}
\begin{UPSTIexercice}{Durée d'une trame DMX complète}
On considère une trame DMX transmettant les \textbf{512 canaux} complets.

Valeurs à utiliser :
\begin{itemize}
    \item BREAK : 176 $\mu$s
    \item MAB : 12 $\mu$s
    \item Durée d'un octet : 44 $\mu$s
\end{itemize}

\UPSTIquestion{Calculer la durée totale de transmission des octets (Start Code + 512 canaux).}
\UPSTIquestion{Calculer la durée totale de la trame complète.}
\UPSTIquestion{Exprimer ce résultat en millisecondes avec 2 chiffres après la virgule.}
\UPSTIquestion{En déduire la fréquence maximale théorique de rafraîchissement (en Hz) pour une trame de 512 canaux.}
\UPSTIquestion{Comparer ce résultat avec la fréquence indiquée dans l'encart page~\pageref{info:refresh}.}
\end{UPSTIexercice}

\begin{UPSTIprofOnlyEnv}
    \begin{UPSTIcorrectionP}{Durée d'une trame DMX complète}
        \textbf{1)} Nombre d'octets : 1 Start Code + 512 canaux = 513 octets
        \[
        T_{\text{octets}} = 513 \times 44~\mu\text{s} = 22\,572~\mu\text{s}
        \]

        \textbf{2)} Durée totale :
        \begin{align*}
        T_{\text{trame}} &= T_{\text{BREAK}} + T_{\text{MAB}} + T_{\text{octets}} \\
        &= 176 + 12 + 22\,572 \\
        &= 22\,760~\mu\text{s}
        \end{align*}

        \textbf{3)} En millisecondes : $T_{\text{trame}} = \boxed{22{,}76~\text{ms}}$

        \textbf{4)} Fréquence maximale :
        \[
        f_{\text{max}} = \frac{1}{T_{\text{trame}}} = \frac{1}{22{,}76 \times 10^{-3}} \approx \boxed{43{,}9~\text{Hz}}
        \]

        \textbf{5)} Ce résultat est cohérent avec la valeur de $\approx$ 44 Hz indiquée page~\pageref{info:refresh}.
    \end{UPSTIcorrectionP}
\end{UPSTIprofOnlyEnv}

\begin{UPSTIexercice}{Optimisation du nombre de canaux}
Dans la pratique, on ne transmet pas toujours les 512 canaux. Un contrôleur DMX peut optimiser la transmission en n'envoyant que les canaux réellement utilisés.

On considère une installation utilisant uniquement les canaux 1 à 24.

\UPSTIquestion{Calculer la durée d'une trame DMX optimisée transmettant uniquement 24 canaux.}
\UPSTIquestion{Calculer la fréquence de rafraîchissement maximale pour cette installation.}
\UPSTIquestion{Quel est le gain de temps (en \%) par rapport à une trame complète de 512 canaux ?}
\end{UPSTIexercice}

\begin{UPSTIprofOnlyEnv}
    \begin{UPSTIcorrectionP}{Optimisation du nombre de canaux}
        \textbf{1)} Durée avec 24 canaux (+ Start Code = 25 octets) :
        \begin{align*}
        T_{\text{octets}} &= 25 \times 44~\mu\text{s} = 1\,100~\mu\text{s} \\
        T_{\text{trame}} &= 176 + 12 + 1\,100 = \boxed{1\,288~\mu\text{s} = 1{,}288~\text{ms}}
        \end{align*}

        \textbf{2)} Fréquence maximale :
        \[
        f_{\text{max}} = \frac{1}{1{,}288 \times 10^{-3}} \approx \boxed{776{,}4~\text{Hz}}
        \]

        \textbf{3)} Gain de temps :
        \[
        \text{Gain} = \frac{22{,}76 - 1{,}288}{22{,}76} \times 100\% \approx \boxed{94{,}3\%}
        \]

        L'optimisation permet une fréquence de rafraîchissement environ \textbf{17 fois plus élevée} !
    \end{UPSTIcorrectionP}
\end{UPSTIprofOnlyEnv}


\subsection{RS-485 et topologie réseau}
\begin{UPSTIexercice}{Caractéristiques du RS-485}
Le protocole DMX512 utilise la couche physique RS-485 pour la transmission.

\UPSTIquestion{Rappeler le principe de la transmission différentielle utilisée par le RS-485.}
\UPSTIquestion{Pourquoi la transmission différentielle offre-t-elle une meilleure immunité au bruit qu'une transmission référencée à la masse ?}
\UPSTIquestion{Quelle est la distance maximale recommandée pour une ligne DMX ?}
\UPSTIquestion{Quelle est la valeur de l'impédance caractéristique du câble DMX ?}
\UPSTIquestion{Pourquoi doit-on placer une résistance de terminaison en bout de ligne ?}
\end{UPSTIexercice}

\begin{UPSTIprofOnlyEnv}
    \begin{UPSTIcorrectionP}{Caractéristiques du RS-485}
        \textbf{1)} La transmission différentielle utilise \textbf{deux fils} (Data+ et Data-). Le signal utile est la \textbf{différence de tension} entre ces deux fils, et non la tension absolue par rapport à la masse.

        \textbf{2)} Avantage : Les perturbations électromagnétiques affectent généralement les deux fils de manière identique (mode commun). La différence de tension reste donc constante, ce qui permet de rejeter le bruit.

        \textbf{3)} Distance maximale : \textbf{300 à 400 mètres} selon les sources (la norme RS-485 permet 400m, mais le DMX est souvent limité à 300m en pratique).

        \textbf{4)} Impédance caractéristique : \textbf{120 $\Omega$}

        \textbf{5)} La résistance de terminaison (120 $\Omega$) évite les \textbf{réflexions de signal} en bout de ligne qui pourraient causer des erreurs de transmission. Elle adapte l'impédance en bout de câble.
    \end{UPSTIcorrectionP}
\end{UPSTIprofOnlyEnv}

\begin{UPSTIexercice}{Topologie et câblage}
\UPSTIquestion{Quelle est la topologie réseau utilisée par le DMX ? (bus, étoile, anneau, chaîne...)}
\UPSTIquestion{Pourquoi ne peut-on pas faire de dérivation en "T" sur un réseau DMX ?}
\UPSTIquestion{Combien de récepteurs maximum peut-on connecter sur un univers DMX ? (limitation électrique)}
\UPSTIquestion{Quelle solution permet de connecter plus de 32 appareils sur un même univers DMX ?}
\UPSTIquestion{Quel type de câble doit-on utiliser pour une installation DMX de qualité ?}
\end{UPSTIexercice}

\begin{UPSTIprofOnlyEnv}
    \begin{UPSTIcorrectionP}{Topologie et câblage}
        \textbf{1)} Topologie : \textbf{Chaîne} (daisy chain) ou bus linéaire.

        \textbf{2)} Les dérivations en "T" créent des \textbf{désadaptations d'impédance} qui provoquent des réflexions de signal et peuvent corrompre les données. Le signal doit traverser chaque appareil séquentiellement (IN $\to$ OUT).

        \textbf{3)} Maximum : \textbf{32 récepteurs} (limitation de charge électrique du RS-485).

        \textbf{4)} Solution : Utiliser un \textbf{splitter DMX} (aussi appelé répéteur ou amplificateur DMX) qui régénère le signal et permet de créer plusieurs branches.

        \textbf{5)} Câble recommandé : \textbf{Paire torsadée blindée} (STP - Shielded Twisted Pair) avec une impédance de 120 $\Omega$.
    \end{UPSTIcorrectionP}
\end{UPSTIprofOnlyEnv}


\subsection{Plan d'adressage complexe}
\begin{UPSTIexercice}{Installation mixte}
On doit installer sur un même univers DMX :
\begin{itemize}
    \item 6 dimmers simples (1 canal chacun)
    \item 4 projecteurs LED RGB (3 canaux chacun)
    \item 2 lyres (16 canaux chacune)
\end{itemize}

On souhaite organiser l'adressage de la manière suivante :
\begin{itemize}
    \item Les dimmers occupent les premières adresses
    \item Les projecteurs RGB suivent
    \item Les lyres en dernier
\end{itemize}

\UPSTIquestion{Calculer le nombre total de canaux utilisés.}
\UPSTIquestion{Vérifier que cette installation est compatible avec un univers DMX (512 canaux).}
\UPSTIquestion{Compléter le plan d'adressage suivant :}

\begin{center}
\begin{tabular}{|c|c|c|c|}
\hline
\textbf{Appareil} & \textbf{Type} & \textbf{Adresse DMX} & \textbf{Canaux utilisés} \\
\hline
Dimmer 1 & 1 canal & 1 & 1 \\
\hline
Dimmer 2 & 1 canal & ? & ? \\
\hline
... & ... & ... & ... \\
\hline
RGB 1 & 3 canaux & ? & ? \\
\hline
... & ... & ... & ... \\
\hline
Lyre 1 & 16 canaux & ? & ? \\
\hline
Lyre 2 & 16 canaux & ? & ? \\
\hline
\end{tabular}
\end{center}

\UPSTIquestion{Quel est le numéro du dernier canal utilisé ?}
\UPSTIquestion{Combien de canaux restent disponibles pour d'autres appareils ?}
\end{UPSTIexercice}

\begin{UPSTIprofOnlyEnv}
    \begin{UPSTIcorrectionP}{Installation mixte}
        \textbf{1)} Nombre total de canaux :
        \begin{align*}
        N_{\text{total}} &= (6 \times 1) + (4 \times 3) + (2 \times 16) \\
        &= 6 + 12 + 32 = \boxed{50~\text{canaux}}
        \end{align*}

        \textbf{2)} $50 < 512$ : \textbf{Compatible} avec un univers DMX.

        \textbf{3)} Plan d'adressage complet :
        \begin{center}
        \begin{tabular}{|c|c|c|c|}
        \hline
        \textbf{Appareil} & \textbf{Type} & \textbf{Adresse DMX} & \textbf{Canaux utilisés} \\
        \hline
        Dimmer 1 & 1 canal & 1 & 1 \\
        Dimmer 2 & 1 canal & 2 & 2 \\
        Dimmer 3 & 1 canal & 3 & 3 \\
        Dimmer 4 & 1 canal & 4 & 4 \\
        Dimmer 5 & 1 canal & 5 & 5 \\
        Dimmer 6 & 1 canal & 6 & 6 \\
        \hline
        RGB 1 & 3 canaux & 7 & 7, 8, 9 \\
        RGB 2 & 3 canaux & 10 & 10, 11, 12 \\
        RGB 3 & 3 canaux & 13 & 13, 14, 15 \\
        RGB 4 & 3 canaux & 16 & 16, 17, 18 \\
        \hline
        Lyre 1 & 16 canaux & 19 & 19 à 34 \\
        Lyre 2 & 16 canaux & 35 & 35 à 50 \\
        \hline
        \end{tabular}
        \end{center}

        \textbf{4)} Dernier canal utilisé : \textbf{50}

        \textbf{5)} Canaux disponibles : $512 - 50 = \boxed{462~\text{canaux}}$
    \end{UPSTIcorrectionP}
\end{UPSTIprofOnlyEnv}

\begin{UPSTIexercice}{Détection d'erreur d'adressage}
Un technicien a mal configuré l'adressage d'une installation comportant :
\begin{itemize}
    \item 1 projecteur RGB à l'adresse 10
    \item 1 stroboscope (1 canal) à l'adresse 11
    \item 1 autre projecteur RGB à l'adresse 13
\end{itemize}

\UPSTIquestion{Identifier le problème dans cette configuration.}
\UPSTIquestion{Expliquer les conséquences sur le fonctionnement des appareils.}
\UPSTIquestion{Proposer une correction de l'adressage pour que tous les appareils fonctionnent correctement.}
\end{UPSTIexercice}

\begin{UPSTIprofOnlyEnv}
    \begin{UPSTIcorrectionP}{Détection d'erreur d'adressage}
        \textbf{1)} Problème : Le premier projecteur RGB (adresse 10) utilise les canaux 10, 11 et 12. Le stroboscope est configuré sur le canal 11, qui est \textbf{déjà utilisé} par le projecteur RGB. Il y a un \textbf{conflit d'adressage}.

        \textbf{2)} Conséquences :
        \begin{itemize}
            \item Le stroboscope et le canal Vert du projecteur RGB écouteront tous deux le canal 11
            \item Le contrôle du projecteur RGB sera incorrect (impossible de régler le vert indépendamment)
            \item Le stroboscope s'activera quand on règle le vert du projecteur
        \end{itemize}

        \textbf{3)} Correction :
        \begin{center}
        \begin{tabular}{|c|c|c|}
        \hline
        \textbf{Appareil} & \textbf{Adresse DMX} & \textbf{Canaux utilisés} \\
        \hline
        Projecteur RGB 1 & 10 & 10, 11, 12 \\
        Stroboscope & 13 & 13 \\
        Projecteur RGB 2 & 14 & 14, 15, 16 \\
        \hline
        \end{tabular}
        \end{center}
    \end{UPSTIcorrectionP}
\end{UPSTIprofOnlyEnv}


\subsection{Connectique DMX}
\begin{UPSTIexercice}{Brochage XLR}
On utilise des connecteurs XLR 5 broches pour une installation DMX professionnelle.

\UPSTIquestion{Donner la fonction de chacune des 5 broches du connecteur XLR 5.}
\UPSTIquestion{Quelles sont les deux broches essentielles pour la transmission du signal DMX ?}
\UPSTIquestion{Pourquoi le connecteur XLR 5 broches est-il préférable au XLR 3 broches pour le DMX ?}
\UPSTIquestion{Peut-on utiliser un câble XLR audio (micro) pour une installation DMX temporaire ? Justifier.}
\end{UPSTIexercice}

\begin{UPSTIprofOnlyEnv}
    \begin{UPSTIcorrectionP}{Brochage XLR}
        \textbf{1)} Brochage XLR 5 broches :
        \begin{itemize}
            \item Broche 1 : Masse / Blindage (GND)
            \item Broche 2 : Data- (DMX-)
            \item Broche 3 : Data+ (DMX+)
            \item Broche 4 : Réservé (seconde paire Data2-)
            \item Broche 5 : Réservé (seconde paire Data2+)
        \end{itemize}

        \textbf{2)} Broches essentielles : \textbf{2 (Data-)} et \textbf{3 (Data+)}

        \textbf{3)} Le XLR 5 broches permet :
        \begin{itemize}
            \item D'éviter la confusion avec les câbles audio XLR 3
            \item De prévoir une seconde paire pour des extensions futures (ex: RDM)
            \item Une meilleure séparation des signaux
        \end{itemize}

        \textbf{4)} En dépannage temporaire, oui, car les broches 1, 2, 3 sont câblées de la même façon. \textbf{MAIS} le câble audio n'a pas la bonne impédance (typiquement 75-110 $\Omega$ au lieu de 120 $\Omega$), ce qui peut causer des problèmes sur de longues distances ou avec beaucoup d'appareils. \textbf{Non recommandé pour une installation permanente}.
    \end{UPSTIcorrectionP}
\end{UPSTIprofOnlyEnv}


\pagebreak

\section{Niveau 3 - Synthèse et applications}

\subsection{Dimensionnement d'une installation complète}
\begin{UPSTIexercice}{Projet d'installation scénique}
Une salle de spectacle souhaite installer un système d'éclairage DMX composé de :
\begin{itemize}
    \item 24 projecteurs PAR LED RGBA (4 canaux chacun)
    \item 8 lyres asservies (20 canaux chacune)
    \item 12 projecteurs à gobo (10 canaux chacun)
    \item 4 machines à fumée (2 canaux chacune)
\end{itemize}

\UPSTIquestion{Calculer le nombre total de canaux DMX nécessaires.}
\UPSTIquestion{Cette installation peut-elle tenir sur un seul univers DMX ? Justifier.}
\UPSTIquestion{Si nécessaire, proposer une répartition des appareils sur plusieurs univers en optimisant l'utilisation des canaux.}
\UPSTIquestion{Établir un plan d'adressage complet pour le premier univers en regroupant les appareils par type.}
\UPSTIquestion{Le câble DMX doit parcourir 180 mètres entre le contrôleur et le dernier appareil. Cette distance est-elle acceptable ? Que proposez-vous si la distance était de 450 mètres ?}
\end{UPSTIexercice}

\begin{UPSTIprofOnlyEnv}
    \begin{UPSTIcorrectionP}{Projet d'installation scénique}
        \textbf{1)} Calcul du nombre de canaux :
        \begin{align*}
        N_{\text{total}} &= (24 \times 4) + (8 \times 20) + (12 \times 10) + (4 \times 2) \\
        &= 96 + 160 + 120 + 8 \\
        &= \boxed{384~\text{canaux}}
        \end{align*}

        \textbf{2)} $384 < 512$ : Oui, cette installation \textbf{peut tenir sur un seul univers DMX}.

        Il reste $512 - 384 = 128$ canaux disponibles.

        \textbf{3)} Un seul univers suffit. Pas besoin de répartition.

        \textbf{4)} Plan d'adressage du premier (et unique) univers :

        \begin{center}
        \scriptsize
        \begin{tabular}{|c|c|c|c|}
        \hline
        \textbf{Appareil} & \textbf{Type} & \textbf{Adresse} & \textbf{Canaux} \\
        \hline
        \multicolumn{4}{|c|}{\textbf{PAR LED RGBA (24 appareils)}} \\
        \hline
        PAR 1 & 4 canaux & 1 & 1-4 \\
        PAR 2 & 4 canaux & 5 & 5-8 \\
        ... & ... & ... & ... \\
        PAR 24 & 4 canaux & 93 & 93-96 \\
        \hline
        \multicolumn{4}{|c|}{\textbf{Lyres asservies (8 appareils)}} \\
        \hline
        Lyre 1 & 20 canaux & 97 & 97-116 \\
        Lyre 2 & 20 canaux & 117 & 117-136 \\
        ... & ... & ... & ... \\
        Lyre 8 & 20 canaux & 237 & 237-256 \\
        \hline
        \multicolumn{4}{|c|}{\textbf{Projecteurs à gobo (12 appareils)}} \\
        \hline
        Gobo 1 & 10 canaux & 257 & 257-266 \\
        Gobo 2 & 10 canaux & 267 & 267-276 \\
        ... & ... & ... & ... \\
        Gobo 12 & 10 canaux & 367 & 367-376 \\
        \hline
        \multicolumn{4}{|c|}{\textbf{Machines à fumée (4 appareils)}} \\
        \hline
        Fumée 1 & 2 canaux & 377 & 377-378 \\
        Fumée 2 & 2 canaux & 379 & 379-380 \\
        Fumée 3 & 2 canaux & 381 & 381-382 \\
        Fumée 4 & 2 canaux & 383 & 383-384 \\
        \hline
        \end{tabular}
        \end{center}

        \textbf{5)} Distance de 180 m : \textbf{Acceptable} (limite DMX : 300-400 m).

        Pour 450 m : \textbf{Hors limite}. Solutions possibles :
        \begin{itemize}
            \item Utiliser un \textbf{répéteur DMX} (splitter) à mi-parcours pour régénérer le signal
            \item Utiliser un système \textbf{DMX sur Ethernet} (Art-Net, sACN) avec conversion DMX/Ethernet
            \item Utiliser une transmission \textbf{DMX sans fil} (wireless DMX)
        \end{itemize}
    \end{UPSTIcorrectionP}
\end{UPSTIprofOnlyEnv}


\subsection{Diagnostic et dépannage}
\begin{UPSTIexercice}{Pannes courantes - Analyse}
Pour chaque situation décrite ci-dessous, identifier la cause probable du problème et proposer une solution.

\textbf{Situation 1 :} Sur une chaîne de 15 projecteurs, seuls les 8 premiers fonctionnent correctement. Les 7 derniers ne répondent pas du tout.

\textbf{Situation 2 :} Les projecteurs en fin de ligne clignotent de manière aléatoire et ne répondent pas correctement aux commandes.

\textbf{Situation 3 :} Tous les appareils d'une installation fonctionnent correctement lorsqu'il y a moins de 10 appareils connectés. Au-delà, des dysfonctionnements apparaissent sur les derniers appareils de la chaîne.

\textbf{Situation 4 :} Un projecteur RGB affiche toujours du rouge à pleine intensité, même quand le contrôleur envoie la valeur 0 sur le canal rouge.

\UPSTIquestion{Pour chaque situation, identifier la cause probable.}
\UPSTIquestion{Proposer une solution ou une procédure de test pour chaque cas.}
\end{UPSTIexercice}

\begin{UPSTIprofOnlyEnv}
    \begin{UPSTIcorrectionP}{Pannes courantes - Analyse}
        \textbf{Situation 1 :}
        \begin{itemize}
            \item \textbf{Cause probable :} Câble coupé ou déconnecté entre l'appareil 8 et l'appareil 9. Ou bien l'appareil 9 a sa sortie DMX OUT défectueuse.
            \item \textbf{Solution :} Vérifier le câble entre les appareils 8 et 9. Tester en connectant l'appareil 9 directement au contrôleur. Si le problème persiste, l'appareil 9 est défectueux (court-circuiter temporairement en reliant 8 à 10).
        \end{itemize}

        \textbf{Situation 2 :}
        \begin{itemize}
            \item \textbf{Cause probable :} Absence de résistance de terminaison en bout de ligne. Les réflexions de signal causent des erreurs de transmission.
            \item \textbf{Solution :} Installer une résistance de 120 $\Omega$ entre Data+ et Data- sur le dernier appareil de la chaîne (ou activer la terminaison intégrée si disponible).
        \end{itemize}

        \textbf{Situation 3 :}
        \begin{itemize}
            \item \textbf{Cause probable :} Dépassement de la limite de 32 récepteurs sur le bus DMX. Au-delà, l'impédance de charge devient trop faible et dégrade le signal.
            \item \textbf{Solution :} Utiliser un splitter DMX pour diviser le bus en plusieurs branches ou régénérer le signal. Alternative : vérifier que certains appareils ne présentent pas une charge d'entrée anormalement élevée.
        \end{itemize}

        \textbf{Situation 4 :}
        \begin{itemize}
            \item \textbf{Cause probable :} Mauvais adressage du projecteur. Il écoute probablement le mauvais canal (conflit d'adresse ou erreur de configuration).
            \item \textbf{Solution :} Vérifier l'adresse DMX configurée sur le projecteur et s'assurer qu'elle correspond au plan d'adressage prévu. Vérifier également le mode de fonctionnement (certains projecteurs ont plusieurs modes utilisant un nombre différent de canaux).
        \end{itemize}
    \end{UPSTIcorrectionP}
\end{UPSTIprofOnlyEnv}


\subsection{Cas pratique de synthèse}
\begin{UPSTIexercice}{Étude de cas : Festival en plein air}
Un organisateur de festival souhaite installer un système d'éclairage pour une scène extérieure. L'installation comprend :

\textbf{Équipements :}
\begin{itemize}
    \item 32 projecteurs PAR LED RGB (3 canaux chacun) répartis sur la structure de scène
    \item 6 lyres (18 canaux chacune) en fond de scène
    \item 8 stroboscopes (1 canal chacun) sur les côtés
    \item 2 lasers (8 canaux chacun)
\end{itemize}

\textbf{Contraintes :}
\begin{itemize}
    \item Le contrôleur DMX est installé en régie, à 85 mètres de la scène
    \item Les projecteurs PAR sont répartis en 4 groupes de 8 projecteurs chacun
    \item Le budget impose l'utilisation d'un seul contrôleur DMX (2 univers disponibles)
\end{itemize}

\UPSTIquestion{Calculer le nombre total de canaux DMX nécessaires.}

\UPSTIquestion{Proposer une répartition des appareils sur 2 univers en justifiant vos choix (facilité de programmation, câblage, etc.).}

\UPSTIquestion{Pour l'univers 1, établir un plan d'adressage détaillé en regroupant les PAR LED par groupe.}

\UPSTIquestion{Calculer la durée d'une trame DMX pour chaque univers (en considérant que seuls les canaux utilisés sont transmis).}

\UPSTIquestion{En déduire la fréquence de rafraîchissement maximale pour chaque univers.}

\UPSTIquestion{La distance de 85 mètres entre la régie et la scène est-elle compatible avec le DMX ? Quelles précautions doit-on prendre ?}

\UPSTIquestion{Proposer un schéma de câblage pour minimiser la longueur de câble nécessaire, sachant que les 4 groupes de PAR sont aux 4 coins de la scène, les lyres en fond de scène et les stroboscopes sur les côtés.}
\end{UPSTIexercice}

\begin{UPSTIprofOnlyEnv}
    \begin{UPSTIcorrectionP}{Étude de cas : Festival en plein air}
        \textbf{1)} Nombre total de canaux :
        \begin{align*}
        N_{\text{total}} &= (32 \times 3) + (6 \times 18) + (8 \times 1) + (2 \times 8) \\
        &= 96 + 108 + 8 + 16 \\
        &= \boxed{228~\text{canaux}}
        \end{align*}

        \textbf{2)} Répartition sur 2 univers (proposition) :

        \textbf{Univers 1 :} Effets statiques et dynamiques (132 canaux)
        \begin{itemize}
            \item 32 PAR LED RGB : $32 \times 3 = 96$ canaux
            \item 6 Lyres : $6 \times 18 = 108$ canaux $\to$ \textbf{Trop ! Mettre 3 lyres seulement = 54 canaux}
            \item Total Univers 1 : $96 + 54 = 150$ canaux
        \end{itemize}

        \textbf{Univers 2 :} Compléments et effets spéciaux (78 canaux)
        \begin{itemize}
            \item 3 Lyres restantes : $3 \times 18 = 54$ canaux
            \item 8 Stroboscopes : $8 \times 1 = 8$ canaux
            \item 2 Lasers : $2 \times 8 = 16$ canaux
            \item Total Univers 2 : $54 + 8 + 16 = 78$ canaux
        \end{itemize}

        \textit{Justification :} Cette répartition sépare les PAR (wash général) des effets mobiles, facilite la programmation et équilibre la charge entre les deux univers.

        \textbf{3)} Plan d'adressage Univers 1 :

        \begin{center}
        \scriptsize
        \begin{tabular}{|c|c|c|}
        \hline
        \textbf{Appareil} & \textbf{Adresse} & \textbf{Canaux} \\
        \hline
        \multicolumn{3}{|c|}{\textbf{Groupe 1 - Jardin avant}} \\
        \hline
        PAR 1 à 8 & 1, 4, 7, ..., 22 & 1-24 \\
        \hline
        \multicolumn{3}{|c|}{\textbf{Groupe 2 - Cour avant}} \\
        \hline
        PAR 9 à 16 & 25, 28, 31, ..., 46 & 25-48 \\
        \hline
        \multicolumn{3}{|c|}{\textbf{Groupe 3 - Jardin arrière}} \\
        \hline
        PAR 17 à 24 & 49, 52, 55, ..., 70 & 49-72 \\
        \hline
        \multicolumn{3}{|c|}{\textbf{Groupe 4 - Cour arrière}} \\
        \hline
        PAR 25 à 32 & 73, 76, 79, ..., 94 & 73-96 \\
        \hline
        \multicolumn{3}{|c|}{\textbf{Lyres (3 premières)}} \\
        \hline
        Lyre 1 & 97 & 97-114 \\
        Lyre 2 & 115 & 115-132 \\
        Lyre 3 & 133 & 133-150 \\
        \hline
        \end{tabular}
        \end{center}

        \textbf{4)} Durée des trames :

        \textit{Univers 1 (150 canaux + Start Code = 151 octets) :}
        \begin{align*}
        T_1 &= 176 + 12 + (151 \times 44) = 188 + 6\,644 = \boxed{6\,832~\mu\text{s} = 6{,}83~\text{ms}}
        \end{align*}

        \textit{Univers 2 (78 canaux + Start Code = 79 octets) :}
        \begin{align*}
        T_2 &= 176 + 12 + (79 \times 44) = 188 + 3\,476 = \boxed{3\,664~\mu\text{s} = 3{,}66~\text{ms}}
        \end{align*}

        \textbf{5)} Fréquences de rafraîchissement :
        \[
        f_1 = \frac{1}{6{,}83 \times 10^{-3}} \approx \boxed{146~\text{Hz}}
        \]
        \[
        f_2 = \frac{1}{3{,}66 \times 10^{-3}} \approx \boxed{273~\text{Hz}}
        \]

        \textbf{6)} Distance de 85 m : \textbf{Compatible} (limite : 300-400 m).

        \textit{Précautions :}
        \begin{itemize}
            \item Utiliser un câble de qualité (paire torsadée blindée, 120 $\Omega$)
            \item Installer une résistance de terminaison en bout de ligne
            \item Protéger le câble des intempéries et du passage du public
            \item Prévoir un chemin de câble dédié, séparé des alimentations électriques 230V
        \end{itemize}

        \textbf{7)} Schéma de câblage proposé :

        \textit{Stratégie :} Créer une chaîne qui minimise les allers-retours.

        Exemple de parcours : Régie $\to$ Groupe 1 (Jardin avant) $\to$ Groupe 3 (Jardin arrière) $\to$ Lyres (fond de scène) $\to$ Groupe 4 (Cour arrière) $\to$ Groupe 2 (Cour avant) $\to$ retour.

        Alternative : Utiliser un splitter DMX au niveau de la scène pour créer plusieurs branches et réduire la longueur totale de câble.
    \end{UPSTIcorrectionP}
\end{UPSTIprofOnlyEnv}


\subsection{Pour aller plus loin}
\begin{UPSTIexercice}{RDM (Remote Device Management)}
Le protocole RDM est une extension du DMX512 qui permet une communication \textbf{bidirectionnelle}.

\UPSTIquestion{Rechercher et expliquer brièvement ce qu'est le protocole RDM.}
\UPSTIquestion{Quels sont les avantages du RDM par rapport au DMX standard ?}
\UPSTIquestion{Quel est le Start Code utilisé pour les messages RDM ?}
\UPSTIquestion{Le RDM est-il rétrocompatible avec le DMX standard ? Expliquer.}
\end{UPSTIexercice}

\begin{UPSTIprofOnlyEnv}
    \begin{UPSTIcorrectionP}{RDM (Remote Device Management)}
        \textbf{1)} Le RDM (ANSI E1.20) est une extension du DMX512 qui ajoute une communication bidirectionnelle. Il permet au contrôleur d'interroger les appareils et de recevoir des informations en retour.

        \textbf{2)} Avantages du RDM :
        \begin{itemize}
            \item Configuration à distance (adressage automatique)
            \item Diagnostic et monitoring (température, heures d'utilisation, erreurs)
            \item Mise à jour de firmware à distance
            \item Récupération d'informations (modèle, numéro de série, modes disponibles)
        \end{itemize}

        \textbf{3)} Le RDM utilise le Start Code \textbf{0xCC} (204 en décimal) pour ses messages.

        \textbf{4)} Oui, le RDM est \textbf{rétrocompatible}. Les trames DMX standard (Start Code 0x00) continuent de fonctionner normalement. Les appareils non-RDM ignorent simplement les trames avec Start Code 0xCC. Les deux types de trames peuvent coexister sur le même bus.
    \end{UPSTIcorrectionP}
\end{UPSTIprofOnlyEnv}

\begin{UPSTIexercice}{DMX sur Ethernet (optionnel)}
Pour les grandes installations, le DMX peut être transporté sur des réseaux Ethernet.

\UPSTIquestion{Citer deux protocoles permettant de transporter du DMX sur Ethernet.}
\UPSTIquestion{Quels sont les avantages de ces solutions par rapport au DMX filaire traditionnel ?}
\UPSTIquestion{Rechercher ce qu'est un "nœud DMX" (DMX node) dans ce contexte.}
\end{UPSTIexercice}

\begin{UPSTIprofOnlyEnv}
    \begin{UPSTIcorrectionP}{DMX sur Ethernet (optionnel)}
        \textbf{1)} Deux protocoles courants :
        \begin{itemize}
            \item \textbf{Art-Net} (protocole ouvert, très répandu)
            \item \textbf{sACN} (streaming ACN, ANSI E1.31, standard ESTA)
        \end{itemize}

        \textbf{2)} Avantages :
        \begin{itemize}
            \item Distribution sur de très longues distances (limitation : réseau Ethernet)
            \item Possibilité de transporter plusieurs univers DMX sur un seul câble
            \item Infrastructure réseau standard (switches, câbles Cat5e/Cat6)
            \item Possibilité de contrôle via réseau IP (WiFi, Internet)
            \item Monitoring et diagnostic via le réseau
        \end{itemize}

        \textbf{3)} Un \textbf{nœud DMX} (DMX node ou gateway) est un appareil qui convertit les données DMX transportées sur Ethernet (Art-Net, sACN) en signaux DMX512 classiques (RS-485). Il possède généralement plusieurs sorties DMX (univers) et une connexion Ethernet.
    \end{UPSTIcorrectionP}
\end{UPSTIprofOnlyEnv}

