% ==================== NIVEAU 1 — VARIABLES / TYPES (10) ====================

\element{calcul}{
  \section{Promotions vestimentaires}
    Un magasin de vêtements propose des promotions pour ses articles. 

  \begin{question}{promo-1}
    Pour ce premier exercice, la réduction est entrée à la main par le directeur du magasin.

    Ecrire un programme \textbf{complet} qui demande le montant total, suivi de la réduction à appliquer \textbf{(en pourcentage)} puis qui indique le montant à payer.
      \evaluationProfGrille{2}{17}
  \end{question}


  \begin{question}{promo-2}
  A présent, les réductions sont automatiques et dépendent du nombre d'articles demandés. 

    \begin{minipage}{.65\textwidth}
      Ecrire un programme qui demande le nombre d'article suivi du montant total puis qui applique les réductions ci-contre.

      \textbf{Pour cette question, écrire uniquement la partie du programme se trouvant à l'intérieur du main. INUTILE d'écrire le préambule.}
    \end{minipage}\hfill
    \begin{minipage}{.3\textwidth}
    \begin{tabular}{cc}
      Nombre d'article n & Promotion \\\hline
      $n \le 2$          & $0\%$\\
      $2 < n \le 5$          & $10\%$\\
      $5 < n \le 8$          & $20\%$\\
    \end{tabular}
    \end{minipage}

      \evaluationProfGrille{2}{17}

  \end{question}

  \begin{question}{promo-3}
    On améliore encore le programme pour qu'il calcule automatiquement le montant total à partir du prix des articles.

    Ecrire les modifications proposées pour que le programme demande les prix des articles un par un, avant d'appliquer la réduction de la question précédente sur le montant total.

    \textit{NB : Au moment du développement du programme, on ne connait pas le nombre d'articles qu'entrera l'utilisateur.}

    \textbf{INUTILE de recopier le code de la question précédente. Indiquer simplement en couleur la portion de code et l'endroit où vous désirer l'insérer. INUTILE d'écrire le préambule également.}

    \evaluationProfGrille{2}{17}
    
  \end{question}
}
